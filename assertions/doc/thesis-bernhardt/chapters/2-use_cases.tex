\chapter{Problem Definition and Use-cases}\label{chap:use_cases}
The introduction argued that validating properties on input parameters becomes expensive when a data structure (such as arrays), certain data types (such as byte sequences) or a range of values have to be iterated. It showed examples for such properties and identified logical formulas containing universal quantification as amenable for distributed assertion checking. As a matter of fact, the same is also valid for existential quantification, even though its negation has a different notion than that of its universal counterpart. The difference between the two is addressed later in this chapter. \\
The following section formally defines a set of applicable logical formulas for the first implementation of this approach. Given this set, section \secref{sec:examples} gives some more in-depth examples including cost analyses. 

\section{Definition of a set of logical formulae}\label{sec:formulae}
\begin{align*}
    t &::= \text{primitive data types (existing in the blockchain VM)} \\
    \sigma &::= \forall \mid \exists \\
    \Theta &::= \Phi \mid \sigma (x:t) \Theta \\
    \Phi,\Psi &::= \neg\Phi \mid \Phi \Rightarrow \Psi \mid \Phi\wedge\Psi \mid
    				\Phi\vee\Psi \mid \Phi \oplus \Psi \mid M \rho N \\
    \rho &::= < \mid > \mid \le \mid \ge \mid = \mid \ne \\
    M, N &::= x \mid c \mid M \odot N  \mid f (\overline M) \\
    \odot &::= +\mid -\mid * \mid / \mid \% \\
    c &::= \text{constants: numbers, strings, } \top, \bot \\
    f &::= \text{operations (existing in the blockchain VM)}
\end{align*}
\begingroup\vspace*{-\baselineskip}
\captionof{figure}{Set of logical formulas amenable to distributed assertion checking (extension of set from \cite{thiemann_2020})}
\vspace*{\baselineskip}\endgroup

The sets of supported data types $t$ and operations $f$ depend on what is supported in the virtual machines of the blockchains. \secref{lala} goes into more detail about what is already supported in Michelson and how the language and VM have to be extended to facilitate the assertion examples given previously and in \secref{sec:example_formulae}.

\section{Examples}\label{sec:examples}\todo{title}
The following examples are excerpts from \cite{bernhardt_veigel_2020}, which contains a more comprehensive collection of use-cases. A assessment of 

\subsection{Two numbers are coprime}\label{subsec:coprime}
The greatest common divisor ($gcd$) of two non-zero integers $a$ and $b$ is the largest integer that divides both evenly \cite{hardy2008introduction}. A more specific $gcd$-problem is finding two numbers which are coprime, that is their greatest common divisor is one, i.e., $gcd(a, b) = 1$ \cite{hardy2008introduction}. To verify if two numbers are coprime, it has to be checked that there exists no $gcd > 1$ , which can be expressed in the following propositional formula:
\begin{equation}\label{eq:coprime-universial}
    (\forall n : int) (2 \le n \le \min(a,b)) \Rightarrow \neg((a \mathbin{\%} n) = 0 \land (b \mathbin{\%} n) = 0)
\end{equation}
Assuming constant time for calculating the two remainders, checking whether two numbers satisfy this property takes linear time, i.e. $\mathcal{O}(n)$, and depends on the size of the smaller number. Applying distributed assertion checking instead, we consider its negation to find a counterexample:
\begin{equation}\label{eq:coprime-existential}
    (\exists n : int) (2 \le n \le \min(a,b)) \land (a \mathbin{\%} n) = 0 \land (b \mathbin{\%} n) = 0
\end{equation}

\subsection{Heap property}
Heaps are data structures based on trees and can appear in two varieties: min and max heaps. Both can be represented as binary trees and have to satisfy the heap property \cite{dict_heap}. In a max heap, any given node has a lower or equal value than the value of its parent. In a min heap, the value is greater or equal than that of its parent \cite{dict_heap_property}. Binary heaps are often implemented as an array, with the root of the tree stored at index 0. The relatives of a node \texttt{k} can be accessed as follows \cite{dict_binary_heap}:
\begin{lstlisting}[language=Solidity, numbers=none, caption=Access a binary heap in array representation]
heap_array[(k-1)/2] // Get parent of node k
heap_array[(k*2)+1] // Get left child of node k
heap_array[(k*2)+2] // Get right child of node k
\end{lstlisting}

The min heap property, for instance, for a binary heap represented in an array $a$, can be expressed in predicate logic with the following formula:
\begin{equation}\label{eq:heap-unversial}
  (\forall k : int) (1 \le k < |a|) \Rightarrow a[\lfloor(k-1)/2)\rfloor] \le a[k]
\end{equation}

The time complexity of checking if $a$ satisfies the heap property is $\mathcal{O}(n)$ in the size of the array if it is dynamically sized, or $\mathcal{O}(1)$ when statically sized. Solidity, for instance, supports both types \cite{solidity_docs}. Similarly to the previous examples, the property of the array is validated by checking for counterexamples by considering its negation:
\begin{equation}\label{eq:heap-unversial-neg}
  (\exists k : int) (1 \le k \le |a|) \land a[\lfloor(k-1)/2)\rfloor] > a[k]
\end{equation}

\section{Existential quantifier and proofs of validity}\label{sec:existential}
The examples given so far in this thesis used universal quantification and required a search for counterexamples in order to check if some assumption about an input parameter is true. However, the same cannot be applied to existential quantifiers. Consider a contract expecting two intersecting sets of integers, represented as arrays. This property can be checked in $\mathcal{O}(|u|*|v|)$. With $n = |u|+|v|$, the worst case for the run time occurs when both sets have the same size, i.e., $|u| = |v| = n/2$. This leads to a quadratic complexity $\mathcal{O}(n/2 * n/2) = \mathcal{O}(n^2/4) = \mathcal{O}(n^2)$.\\
The assumption, that the intersection of the sets is not empty, i.e., $u \cap v \neq \emptyset$, can be expressed by
\begin{equation}\label{eq:intersect}
  (\exists i, \exists j) (0 \le i < |u|) \land (0 \le j < |v|) \Rightarrow u[i] = v[j]
\end{equation}
and its negation is
\begin{equation}\label{eq:intersect_neg}
  (\forall i, \forall j) (0 \le i < |u|) \land (0 \le j < |v|) \land u[i] \neq v[j]
\end{equation}
If the assertion `fails' for random values of $i,j$, it does not produce a counterexample, but it rather produces a proof for the validity of the assumption. In that case, the validator finding such a proof should publish an approval instead of a veto to the network and the normal execution of the contract can proceed. Hence, there must be two different modi of assertion checking that implement different types of assertions.
\todo{Refer to a chapter where this is discussed further?}

\section{Nested quantification}
The previous example showed that universal and existential quantification have to be handled differently. However, other special cases caused by nested quantification have to be considered as well. Trivial cases of nested quantification are nesting of the same quantifiers, e.g. $(\forall i, j)$. They can be handled by generating a random value for each predicate and checking for a counterexample or proof respectively. It becomes less trivial though, if the nesting contains both universal and existential quantifiers. This section examines the the case of $\exists i, \forall j$ on the basis of a contract that expects a solution for a boolean satisfiability problem (SAT). Another instance of this use-case is also used to show that properties expressed with the nesting $\forall i, \exists j$ cannot be checked with the proposed approach for distributed assertion checking.

\subsection{Boolean satisfiability problem}\label{sec:sat}
Instances of SAT problems consist of a formula $F$ in propositional logic, and determine if there exists a variable assignment $\mathcal{A}$ to the values $true$ or $false$, s.t. the formula is satisfied ($\mathcal{A}(F) = 1$) \cite{Biere2009}. Modern SAT-solvers expect $F$ to be in conjunctive normal form (CNF), i.e., a conjunction of a set of clauses, where clauses are disjunctions of variables or their negations (called literals) \cite{cnf_math_encycl}. Verifying whether a given assignment satisfies the formula takes $\mathcal{O}(n*m)$, where $n$ is the number of clauses and $m$ the number of variables (in the worst case, every variable is present as a literal in each clause). In the case of k-SAT problems, where each clause contains at most k literals \cite{Biere2009}, the complexity is reduced to $\mathcal{O}(n)$.

\subsubsection{Encoding of variables, literals and clauses}
A SAT problem could be represented as a list of clauses $c$ and a variable assignment $a$. To avoid having to handle the variables and their literals as strings and thus optimize runtime and memory consumption, an encoding scheme taken from this blogpost\cite{sabablog} is used: \\
Variables are encoded by a unique number starting from $0$ to $n-1$, where $n$ is the total number of variables. The positive and negative literals of a variable with id $x$ are encoded by the function
\begin{align*}
encode(x) =
\begin{cases}
  $2x$  & \text{if x is present as a positive literal}\\
  $2x+1$ & \text{if x is present as a negated literal}\\
\end{cases}   
\end{align*}

Consequently, checking whether a literal $y$ is positive or negative is a simple bit-wise $AND$ operation:
\begin{align*}
\texttt{isPositive(y) = y \& 1 == 0}
\end{align*}

As already mentioned, $F$ can be represented as a list of clauses, and each clause in turn can be represented as a list of literals. Consider the following boolean formula and variable assignment:
\begin{align*}
F = (A \vee B) \wedge (B \vee \neg C) \text{, } \mathcal{A} = \{A \rightarrow false, B \rightarrow true, C \rightarrow false\}
\end{align*}

Following the encoding of variables given above, the list of variables $v=[A,B,C]$ occurring in $F$ is encoded as $v_{enc} = [0,1,2]$. $F$ is thus represented by the encoding $F_{enc} = [[0,2],[2,5]]$. Since the encoding of variables corresponds to their index in the list of variables, $\mathcal{A}$ can be passed as a list of booleans that maintains the same order of variables: $\mathcal{A}_{enc} = [False, True, False]$. If the blockchains VM supports a respective data type, $\mathcal{A}$ can alternatively be represented with a more efficient bitarray. Lastly, the variable encoding $x$ of a literal $y$ is obtained by dividing its encoding by two, which corresponds to a bit-wise shift to the right:
\begin{align*}
\texttt{decode(y) = y >> 1}
\end{align*}

\subsection{Satisfiability in predicate logic - DNF}\label{sec:dnf}
In opposition to the statement that modern SAT solvers expect $F$ to be in CNF, consider a contract that expects a SAT problem in disjunctive normal form (DNF) and a proposed assignment, both encoded as described above. A boolean formula in DNF is a disjunction of conjunctions \cite{dnf_math_encycl}. Assuming that $F$ is a k-SAT instance, the assumption that $\mathcal{A}$ satisfies $F$ can be expressed in predicate logic as follows:
\begin{gather*}\label{eq:dnf_sat}
\begin{aligned}
(\exists i : int, \forall j : int) (0 \leq i < |F|) \wedge (0 \leq j < k) &\Rightarrow ((isPositive(F[i][j]) \wedge \mathcal{A}[decode(F[i][j])]) \\
&\vee (\neg isPositive(F[i][j]) \wedge \neg \mathcal{A}[decode(F[i][j])])))
\end{aligned}
\end{gather*}
With single quantification, we so far considered the negation of the formula in order to identify counterexamples (or proofs). However, by checking some literal in some clause, the validator cannot make a statement about the validity of the property, as the clauses have to be examined as a whole in order to verify that the conjunction evaluates to true. Consequently, the distribution can only take place on the level of clauses, but not on the level of literals. Each validator thus checks all literals of one random clause and produces a proof of validity if all of them satisfy the property. This implies that the formula does not have to be negated. \\
Besides a distinct transformation, this case also has to be handled differently during the compilation process; instead of translating both quantifiers into a random generator, the universal quantifier has to be translated into a loop, cf. \lstref{lst:rand_loop}.
\begin{lstlisting}[label=lst:rand_loop]
i = random(0, size(F))
for (j : [0..k)):
	check(...)
\end{lstlisting}

Concerning the costs of checking such a formula, it's obvious that loops consume more gas than checking single elements. In the case of k-SAT problems with small values for $k$, the constant time factor for checking a clause might still be considered acceptable.

\subsection{Satisfiability in predicate logic - CNF}
Unfortunately, the same cannot be applied for the reverse quantification nesting. Consider the same contract, but instead of expecting $F$ in DNF, it expects it to be in CNF. The domain restrictions and open formula remain the same, but the quantifiers are in reverse order:
\begin{gather*}\label{eq:cnf_sat}
\begin{aligned}
(\forall i : int, \exists j : int) (0 \leq i < |F|) \wedge (0 \leq j < k) &\Rightarrow ((isPositive(F[i][j]) \wedge \mathcal{A}[decode(F[i][j])]) \\
&\vee (\neg isPositive(F[i][j]) \wedge \neg \mathcal{A}[decode(F[i][j])])))
\end{aligned}
\end{gather*}

Again, the validator has to check the whole clause in order to know whether it evaluates to true. Due to existential quantification, however, the literals in the clauses cannot be checked independent of each other. A single literal evaluating to false does not make the whole clause evaluating to false. Although technically possible, it requires a more sophisticated compiler that translates this nesting into the following code structure:
\begin{lstlisting}[label=lst:rand_loop_dependent]
i = random(0, size(F))
for (j : [0..k)):
	if check(...):
		assert(true)
assert(false)
\end{lstlisting}

\subsection{Further nesting}
Nesting is, of course, not limited to two quantifiers, but can contain many of them. Generally, formulas containing no quantification alterations benefit the most from distributed assertion checking, as each quantifier can be translated into a random generator. On the other hand, formulas with a quantifier alteration early in the order of quantifiers benefit the least. Starting from the first alteration, the quantifiers are translated to loops.\\
To substantiate this argument, consider a two variations of a contract that expect a list of boolean formulas in DNF, and an assignment $\mathcal{A}$. The first one checks, whether $\mathcal{A}$ satisfies at least one $F$, the other one checks if all $F$ are satisfied. This can be expressed by nesting quantifications as follows: 
\begin{itemize}
\item $\exists f \exists i \forall j$
\item $\forall f \exists i \forall j$
\end{itemize}
In the first case, each validator can pick a random clause in a random $F$ and check, if all literals evaluate to true. In the second case, the validators have to find a counterexample, namely a clause that is not satisfied by $\mathcal{A}$. Checking a random clause in isolation cannot disprove the satisfiability, hence the validator has to check all clauses.

\subsection{Restriction of use cases} %name?
The examples given in this chapter have shown that, depending on the type and order of quantification, not all properties can be checked in the same way and need different modi of assertion checking. This increases the complexity and requirements for a implementation of this approach; therefore, for a first iteration, this thesis focuses on providing the fundamentals and implementation for assertions using universal quantification.
\todo{Refer to section where proofs are picked up again?}

\section{Cost analysis} \todo{this}
% diagram cost increase?
% experiment: write Liquidity contract w. 1 loop and 1 single execution * loop size
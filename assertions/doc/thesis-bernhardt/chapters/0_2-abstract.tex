\chapter*{Abstract}
Modern blockchains provide a platform for building sophisticated applications with smart contracts. Because transactions are processed at every node in the network, the quantity and complexity of computations, which can be performed by smart contracts, are limited. Furthermore, complex contracts can generate high transaction costs for the user. To overcome this limitation and improve cost efficiency, blockchain-based applications could move computations to an off-chain component and only process the result on the blockchain. This thesis proposes a processing model for the Tezos blockchain, which engages the validators of the network in a distributed effort to verify a submitted off-chain computation. Based on a specification of properties, the validators are challenged to find a counterexample by asserting these properties for independent subsets of the result. This reduces the computational effort on the blockchain to the verification of the result, rather than the computation of the result itself. Moreover, by implementing a distributed scheme, each node only processes a fraction of the verification. For this processing model, this thesis contributes a domain-specific language to express explicit assertions for off-chain computations and specifies a processing pipeline to transform and compile these assertions into smart contracts. These contracts conduct a point-wise search for a counterexample and are called during the verification mechanism. The provided proof-of-concept implementation of this pipeline targets Tezos, but is by design extendible for other blockchains supporting smart contracts.
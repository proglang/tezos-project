\chapter*{Abstract}
Modern blockchains provide a platform for building applications with Smart Contracts. Because transactions are processed at every node in the network, the quantity and complexity of computation, which can be performed by Smart Contracts, is limited. Furthermore, complex contracts can generate high transaction costs for the user. To overcome this limitation and increase cost efficiency, blockchain-based applications could move computations to an off-chain component and only process the result on the blockchain.
This thesis proposes a processing model, which engages the validators of the network in a distributed effort to verify a submitted off-chain computation. Based on a specification of properties, the validators are challenged to find a counterexample by asserting these properties for independent subsets of the result. This reduces the computational effort on the blockchain to the verification of the result, instead of the computation of the result itself. Due to the distributed scheme, the computations per node are decreased even further. As a further contribution, this thesis defines a domain-specific language to express explicit assertions for off-chain computations. It describes the model and a proof-of-concept implementation of a processing pipeline, which transforms and compiles these assertions into Smart Contracts for the target platform Tezos. These contracts conduct a point-wise search for a counterexample and are called during the verification process.
\chapter{Conclusion and outlook}\label{chap:conclusion}
Complex blockchain-based applications suffer from high computation costs and limited scalability. A consequential measure to address this issue, is to move computations off the chain and only verify the correctness of submitted results on the blockchain. This thesis  proposed assertions for distributed refutation as a probabilistic implementation of this verification. In the processing model of this approach, the validators of the system are challenged to assert specified properties for a random subset of the off-chain computation. If an assertion fails, the respective validator generates a counterexample and publishes it to the network. When the system acknowledges the counterexample, the off-chain computation is rejected. Suitable use-cases for this approach are computations, whose properties can be specified with formulas in predicate logic featuring universal quantification. To support this model on the target platform Tezos, a toolchain was developed to allow developers to specify explicit assertions and generate corresponding Smart Contracts. These contracts check the property point-wise within the quantification domain and identify counterexamples when executed by the validators. For the specification of assertions, a simple domain-specific language was defined.

For the proposed architecture, the conducted costs analysis shows high transaction fees for the submission of off-chain computations,  due to high verification costs. Although the verification might still be more efficient than the execution itself, it can be assumed that the cost reductions are equalised in cases of moderate complexity of off-chain computations. Inter-contract communication was identified as a main driving factor of the verification cost. To lower these costs and approach usability on production systems, the thesis provided possible improvements and modifications of the architecture.

In case the verification cost can be lowered to a feasible threshold, this approach provides a simple solution for increasing the scalability of blockchain-based applications. Assertions can be specified in an expressive and familiar way, and the processing model only requires moderate adaptions to the blockchain protocol.

Based on either the proposed or a modification of the architecture, subsequent tasks include the implementation of the compiler in order to conclude the toolchain. As addressed in the introduction, the implementation of distributed assertions on blockchains also requires a design and implementation of a protocol amendment. Many of the aspects that need to be considered by the protocol design have already been pointed out by this work. This includes, i.a., the extensions to Michelson and Tezos' VM to support random value generation, the implementation of a dedicated execution mode, the withholding of the transaction submitting an off-chain computation until it has been verified, and the rejection of this transaction as a consequence of a verified counterexample. 

Besides the completion of the proof-of-concept implementation, the assertion language can be extended with features, such as variables and user-defined functions, to improve (re-)usability. Furthermore, an interesting extension is the support for properties formulated with existential quantification, as this allows to verify off-chain computations by distributed validation in addition to distributed refutation. 

Current work at the chair of programming languages includes another thesis to develop a model and an implementation of distributed assertion checking targeting the blockchain Ethereum. The groundwork provided in this thesis, such as the use-case analysis, the domain-specific language and frontend of the toolchain, is by design compatible for both targets.
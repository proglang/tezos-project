\chapter{Conclusion and outlook}\label{chap:conclusion}
In order to make use of off-chain computations in smart contracts, this thesis proposed an approach that engages the validators of the blockchain network in a distributed effort to check assertions over input parameters. It covered the partially generic off-chain component of the design and implementation of this approach, while also providing some projections and considerations for the Tezos specific protocol design.

As a first step, it identified properties that can be checked with the proposed scheme, represented by a set of formulas in propositional logic using quantification. By means of some practical use-cases, the semantic differences between universal and existential quantification in relation to assertion checking were highlighted, namely that existential quantification requires the validators to find proofs instead of counterexamples. Even though the thesis focused on distributed refutation, distributed verification is a meaningful extension of future work on this topic.

Based on the set of formulas, the generic off-chain design introduced a simple syntax to express assertions for one or more entrypoints of a contract, followed by a formalization of their transformation which comprises the negation of the formula and translation of quantifiers to efficient random generators. After exploring several orchestration strategies of contract and assertion code, the Tezos specific off-chain design proposed to originate the assertion and manager code modularly and discussed the necessary implementations within the off-chain infrastructure and, partially, the protocol design. This off-chain infrastructure was implemented in OCaml in form of a toolchain, which translates a file of assertions to executable smart contracts for the target platform.\\


% paragraph evaluation

%In conclusion, it analysed the cost incurred by checking assertions as proposed and compared the results to a non-distributed approach. Based on these results, thesis discusses alternatives or possible modifications to solve some of the issues at hand. \\

%contains a compilation pipeline written in OCaml that


%An analysis of the general accuracy of the proposed scheme showed that 
% and stating a formula to compute the minimum amount of random checks to reach a certain confidence level,
%- high costs due to several reasons: random checking values require more test runs than size of iteration space. going below n checks does not provide a high certainty over validity. This approach is thus not suitable for safety or security critical applications. 
%- more effort in compilation returns a better cost-performance ratio if the manager contract and assertions are merged. Still, more base costs
%- depends on the protocol design, whether this approach is worth pursuing -> if the cost for validation can be eliminated by using rewards, it might be realisable.
%- quantifiers: other semantics for certain combinations/existentials --> proofs; here not considered, but would be interesting for future work on this topic

\subsection{Outlook}

\draft{}
\begin{itemize}
\item Summarize what has been implemented so far
\item Next steps
	\begin{itemize}
	\item extending VM \& Michelson
	\item compiler
	\item list open questions
	\item Protocol-design
		\begin{itemize}
		\item implement extensions -> rand instruction only available during assertion checking
		\item Do endorsers validate or extra validators?
		\item How to differentiate between SC w/ and w/o assertions?
		\item How to implement the waiting?
		\item Which transactions/messages can be reused or need to be introduced
		\item Representing counterexamples and verifying them
		\item Penalty for incorrect counterexamples
		\end{itemize}
	\end{itemize}
\item Estimation of cost reduction (or increase?)
\item Alternatives (e.g. TrueBit)
\end{itemize}
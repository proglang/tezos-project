\section{Formal Definitions}
\label{sec:formal-definitions}
\begin{figure} []
\begin{align*}
T, U &::= \\
   &\Mid\ \langle \text{comparable type}\rangle \\
   &\Mid\ \text{option}\ \langle\text{type}\rangle \\
   &\Mid\ \text{list}\ \langle\text{type}\rangle \\
   &\Mid\ \text{set}\ \langle\text{comparable type}\rangle \\
   &\Mid\ \text{operation} \\
   &\Mid\ \text{contract}\ \langle\text{type}\rangle \\
   &\Mid\ \text{ticket}\ \langle\text{comparable type}\rangle \\
   &\Mid\ \text{pair}\ \langle\text{type}\rangle \langle\text{type}\rangle \\
   &\Mid\ \text{or}\ \langle\text{type}\rangle \langle\text{type}\rangle \\
   &\Mid\ \text{lambda}\ \langle\text{type}\rangle \langle\text{type}\rangle \\
   &\Mid\ \text{map}\ \langle\text{comparable type}\rangle \langle\text{type}\rangle \\
   &\Mid\ \text{big-map}\ \langle\text{comparable type}\rangle \langle\text{type}\rangle \\
   &\Mid\ \text{bls12-381-g1} \\
   &\Mid\ \text{bls12-381-g2} \\
   &\Mid\ \text{bls12-381-fr} \\
   &\Mid\ \text{sapling-transaction}\ \langle\text{natural number constant}\rangle \\
   &\Mid\ \text{sapling-state}\ \langle\text{natural number constant}\rangle \\
   &\Mid\ \text{chest} \\
   &\Mid\ \text{chest-key} \\
\langle\text{comparable type}\rangle ::= \\
   &\Mid\ \text{unit} \\
   &\Mid\ \text{never} \\
   &\Mid\ \text{bool} \\
   &\Mid\ \text{int}\\
   &\Mid\ \text{nat}\\
   &\Mid\ \text{string}\\
   &\Mid\ \text{chain-id}\\
   &\Mid\ \text{bytes}\\
   &\Mid\ \text{mutez}\\
   &\Mid\ \text{key-hash}\\
   &\Mid\ \text{key}\\
   &\Mid\ \text{signature}\\
   &\Mid\ \text{timestamp}\\
   &\Mid\ \text{address}\\
   &\Mid\ \text{tx-rollup-l2-address}\\
   &\Mid\ \text{option}\ \langle\text{comparable type}\rangle\\
   &\Mid\ \text{or}\ \langle\text{comparable type}\rangle \langle\text{comparable type}\rangle\\
   &\Mid\ \text{pair}\ \langle\text{comparable type}\rangle \langle\text{comparable type}\rangle \DOT \\
\end{align*}
\caption{Types}
\label{fig:type}
\end{figure}

\begin{figure}[h]
\begin{align*}
\text{t} &::= \\
   &\Mid\ \langle \text{variable} \rangle \\
   &\Mid\ \langle \text{account constant} \rangle \\
   &\Mid\ \langle \text{int constant} \rangle \\
   &\Mid\ \langle \text{string constant} \rangle \\
   &\Mid\ \langle \text{byte sequence constant} \rangle \\
   &\Mid\ \UNIT \\
   &\Mid\ \TRUE \\
   &\Mid\ \FALSE \\
   &\Mid\ \PAIR\ \text{t1 t2}\\
   &\Mid\ \LEFT\ \text{t}\\
   &\Mid\ \RIGHT\ \text{t}\\ 
   &\Mid\ \SOME\ \text{t}\\
   &\Mid\ \NONE \\
   &\Mid\ \text{\{t ; ... \}}\\
   &\Mid\ \text{\{ Elt t1 t2 ; ... \}}\\
   &\Mid\ \{ \langle\text{instruction}\rangle; ... \}   \\
\langle \text{variable} \rangle &::= \\ 
   &\Mid\ \VAR\\
\langle \text{account constant} \rangle &::= \\ 
   &\Mid\ \text{balance} \\
   &\Mid\ \text{amount} \\
   &\Mid\ \text{sender} \\
   &\Mid\ \text{source} \\
   &\Mid\ \text{now} \\
   &\Mid\ \text{level} \\
   &\Mid\ \text{chain-id} \\
   &\Mid\ \text{self}  \\
   &\Mid\ \text{self-address}  \\
   &\Mid\ \text{total-voting-power}  \\
   &\Mid\ \text{voting-power}  \\
\langle \text{natural number constant} \rangle &::= \\ 
   &\Mid\ \text {[0-9]+} \\
\langle \text{int constant} \rangle &::= \\
  &\Mid\ \langle \text{natural number constant} \rangle \\
  &\Mid\ \text{-} \langle \text{natural number constant} \rangle \\
\langle\text{string constant}\rangle &::= \\
  &\Mid\ \text{"} \langle \text{string content}\rangle\text{*"} \\
\langle\text{instruction}\rangle &::= \\
  &\Mid\ \DROP \\
  &\Mid\ \DROP \langle\text{natural number constant}\rangle \\
  &\text{...}
\end{align*}
\caption{Terms}
\label{fig:term}
\end{figure}



\subsubsection{Basic Instructions}

\paragraph{Control structures}
%EXE
\begin{mathpar}
  \inferrule[EXEC]
  {
    [\INSTRUCTIONONE, (\StackOne, \TYF) \STACKCONCAT \EMPTYSTACK, 
    Q]
    \StateTrans^*
    [ \EMPTYSTACK, (\StackOne', \TYS) \STACKCONCAT Q']
  }{
     [(\EXEC; \INSTRUCTION),   (\{\INSTRUCTIONONE\}, \TYF\ \rightarrow\ \TYS) \STACKCONCAT
    (\StackOne, \TYF) \STACKCONCAT \STACK, \PREDICATE \wedge Q] 
    \StateTrans \
    [ \INSTRUCTION, (\StackOne', \TYS) \STACKCONCAT \STACK,
    \PREDICATE \wedge Q']}
\end{mathpar}

%APPLY
\begin{mathpar}
  \inferrule[APPLY]
  {
  }{
    [(\APPLY; \INSTRUCTION), (\StackOne, \TYF) \STACKCONCAT(\{\INSTRUCTIONONE\}, \TLAMBDA\ (\TPAIR\ \TYF\  \TYS)\ \TYT) \STACKCONCAT\STACK, \PREDICATE] \StateTrans\\ [\INSTRUCTION, (\{\PUSH\ \TYF\ \StackOne; \IPAIR; \INSTRUCTIONONE\}, \TLAMBDA\ \TYS\ \TYT ) \STACKCONCAT\STACK, \PREDICATE]}
\end{mathpar}

%LAMBLA
\begin{mathpar}
  \inferrule[LAMBDA]
  {  
  }{
    [(\LAMBDA\ \TYF\ \TYS\ \{ \INSTRUCTIONONE \} ; \INSTRUCTION),\STACK, \PREDICATE] \StateTrans\ [\INSTRUCTION, (\{\INSTRUCTIONONE\}, \TLAMBDA\ \TYF\ \rightarrow\ \TYS) \STACKCONCAT\STACK, \PREDICATE]}
\end{mathpar}

\paragraph{Stack Manipulation}
%DIG
\begin{mathpar}
\inferrule[DIG]
  {
   \FLEN(\A) \EQ\ \N
  }
  {[(\DIG\ \N ; \INSTRUCTION), \A\ \At\ (\StackOne, \TY) \STACKCONCAT\B, \PREDICATE] \StateTrans 
[\INSTRUCTION, (\StackOne, \TY) \STACKCONCAT\A\ \At\ \B, \PREDICATE]}
\end{mathpar}

%DIP
\begin{mathpar}
  \inferrule[DIP]
  {
    [\INSTRUCTIONONE,  \STACK, Q]
    \StateTrans^*
    [\EMPTYSTACK,  \STACK_1, Q']
  }
  {[(\DIP\ \INSTRUCTIONONE; \INSTRUCTION), (\StackOne, \TY) \STACKCONCAT
    \STACK, \PREDICATE \wedge Q]
    \StateTrans 
    [\INSTRUCTION, (\StackOne, \TY) \STACKCONCAT\STACK_1,
    \PREDICATE \wedge Q']
  }
\end{mathpar}

%DIP n
\begin{mathpar}
\inferrule[DIP n]
  { 
     \FLEN(\A) \EQ\ \N \\ [\INSTRUCTIONONE,  \B, Q]
    \StateTrans^*
    [\EMPTYSTACK,  \B_1, Q']
  }
  {[(\DIP\ \N\ \INSTRUCTIONONE; \INSTRUCTION), \A\ \At\ \B, \PREDICATE \wedge Q] \StateTrans 
[(\INSTRUCTION), \A\ \At\ \B_1, \PREDICATE \wedge Q']}
\end{mathpar}

%PUSH
\begin{mathpar}
  \inferrule[PUSH]
  {  
  }{
    [(\PUSH\ \TY\ \X; \INSTRUCTION),\STACK, \PREDICATE] \StateTrans\ [\INSTRUCTION, (\X, \TY) \STACKCONCAT\STACK, \PREDICATE]}
\end{mathpar}


\paragraph{Arithmetic operations}
%ADD
\begin{mathpar}
\inferrule[ADD]
  {
  }
  {[(\ADD; \INSTRUCTION), (\StackOne, \TNAT) \STACKCONCAT(\StackTwo, \TNAT) \STACKCONCAT\STACK, \PREDICATE] \StateTrans \
[\INSTRUCTION, (\X, \TNAT) \STACKCONCAT\STACK, \PREDICATE \Wedge\ (\X\ \EQ\ \StackOne\ \PLUS\ \StackTwo)]}
\end{mathpar}

%ABS
\begin{mathpar}
\inferrule[ABS]
  {
  }
  {
    [(\ABS; \INSTRUCTION), (\StackOne, \TINT) \STACKCONCAT \STACK,
    \PREDICATE]
    \StateTrans \
    [\INSTRUCTION, (\X, \TNAT) \STACKCONCAT \STACK,
    \PREDICATE \wedge (\StackOne \ge 0 \Rightarrow \X =
    \StackOne) \wedge (\StackOne <0 \Rightarrow \X = -\StackOne)]
 }
\end{mathpar}

% COMPARE
\begin{mathpar}
\inferrule[COMPARE-nat]
  {
  }
  {
    [(\COMPARE ; \INSTRUCTION), (\StackOne, \TNAT) \STACKCONCAT (\StackTwo, \TNAT)
    \STACKCONCAT \STACK, \PREDICATE ]
    \SystemTrans \\
    [\INSTRUCTION, (\X, \TINT) \STACKCONCAT \STACK, \PREDICATE
    \wedge\ (\StackOne > \StackTwo \Leftrightarrow \X = 1)
    \wedge\ (\StackOne = \StackTwo \Leftrightarrow \X = 0) 
    \wedge\ (\StackOne < \StackTwo \Leftrightarrow \X = -1)]
    }
\end{mathpar}

\begin{mathpar}
\inferrule[COMPARE-some-some]
  {
  [\COMPARE,  (\X, \TY) \STACKCONCAT (\Y, \TY) \STACKCONCAT\EMPTYSTACK, Q]
    \StateTrans^*
    [\EMPTYSTACK,  (\VariableA, \TINT) \STACKCONCAT\EMPTYSTACK, Q']
  }
  {
    [(\COMPARE ; \INSTRUCTION), (\StackOne, \TOPTION\ \TY) \STACKCONCAT (\StackTwo, \TOPTION\ \TY)
    \STACKCONCAT \STACK, \PREDICATE \wedge Q]
    \SystemTrans \\
    [\INSTRUCTION, (\VariableA, \TINT) \STACKCONCAT \STACK,  \PREDICATE
    \wedge (\StackOne\ \EQ\ \SOME\ \X)
    \wedge (\StackTwo\ \EQ\ \SOME\ \Y)
    \wedge Q']
    }
\end{mathpar}

\begin{mathpar}
\inferrule[COMPARE-some-none]
  {
  }
  {
    [(\COMPARE ; \INSTRUCTION), (\StackOne, \TOPTION\ \TY) \STACKCONCAT (\StackTwo, \TOPTION\ \TY)
    \STACKCONCAT \STACK, \PREDICATE]
    \SystemTrans \\
    [\INSTRUCTION, (1, \TINT) \STACKCONCAT \STACK,  \PREDICATE
    \wedge (\StackOne\ \EQ\ \SOME\ \X)
    \wedge (\StackTwo\ \EQ\ \NONE)
    ]
    }
\end{mathpar}

\paragraph{Boolean operations}
%XOR
\begin{mathpar}
\inferrule[XOR]
  {
  }
  {[(\XOR; \INSTRUCTION), (\StackOne, \TBOOL) \STACKCONCAT(\StackTwo, \TBOOL) \STACKCONCAT\STACK, \PREDICATE] \StateTrans \
[\INSTRUCTION, (\X, \TBOOL) \STACKCONCAT\STACK, \PREDICATE \Wedge\ (\X\ \EQ\ \StackOne\ \FXOR\ \StackTwo)]}
\end{mathpar}

\paragraph{Crytographic oprerations}
%HASH-KEY
\begin{mathpar}
\inferrule[HASH-KEY]
  {
  }
  {[(\HASHKEY; \INSTRUCTION), (\StackOne, \TBYTE) \STACKCONCAT\STACK, \PREDICATE] \StateTrans \
[\INSTRUCTION, (\X, \TBYTE) \STACKCONCAT\STACK, \PREDICATE \Wedge\ (\X\ = \FHASHKEY(\StackOne))]}
\end{mathpar}

%AMOUNT
\paragraph{Blockchain operations}
\begin{mathpar}
\inferrule[AMOUNT]
  {
  }
  {[(\AMOUNT; \INSTRUCTION), \STACK, \PREDICATE] \StateTrans \
[\INSTRUCTION, (\CAMOUNT, \TMUTEZ) \STACKCONCAT\STACK, \PREDICATE}
\end{mathpar}

%CONTRACT ty
\begin{mathpar}
\inferrule[CONTRACT TY - some]
  {
  }
  {[(\CONTRACT\ \TY ; \INSTRUCTION), (\StackOne, \TADDR) \STACKCONCAT\STACK, \PREDICATE] \SystemTrans \
[\INSTRUCTION, (\SOME\ \X, \TOPTION\ (\TCONTRACT\ \TY)) \STACKCONCAT\STACK, \\ \PREDICATE \Wedge\ (\GETCONTRACTTYPE(\StackOne, \TY) \EQ\ \SOME\ \X)]}
\end{mathpar}

\begin{mathpar}
\inferrule[CONTRACT TY - none]
  {
  }
  {[(\CONTRACT\ \TY ; \INSTRUCTION), (\StackOne, \TADDR) \STACKCONCAT\STACK, \PREDICATE] \SystemTrans \
[\INSTRUCTION, \NONE \STACKCONCAT\STACK, \PREDICATE \Wedge\ (\GETCONTRACTTYPE(\StackOne, \TY) = \NONE]}
\end{mathpar}

\paragraph{Operations on data structures}
%CAR
\begin{mathpar}
\inferrule[\CAR]
  {
  }
  {[(\CAR; \INSTRUCTION), (\StackOne, \TPAIR\ \TYF\ \TYS) \STACKCONCAT\STACK, \PREDICATE] \StateTrans \
[\INSTRUCTION, (\X, \TYF) \STACKCONCAT\STACK, \PREDICATE\ \Wedge\ (\StackOne\ \EQ\ \PAIR\ \X\ \Y)]}
\end{mathpar}

\subsubsection{Branch Instructions}
%IF

\begin{mathpar}
  \inferrule[IF]
  {  
  }{
    \{[(\IF\ \INSTRUCTIONONE\  \INSTRUCTIONTWO; \INSTRUCTION),
    (\StackOne, \TBOOL) \STACKCONCAT\STACK, \PREDICATE]\} \cup \SYSTEM 
      \SystemTrans 
    \{[\INSTRUCTIONONE, \STACK, \PREDICATE\ \Wedge\ \StackOne]\} \cup \{ [\INSTRUCTIONTWO, \STACK, \PREDICATE\ \Wedge\ \NEG\
   \StackOne]\} \cup  \SYSTEM
  }
\end{mathpar}

The \IFLEFT\ \INSTRUCTIONONE\  \INSTRUCTIONTWO\ instruction expects a or value at the top of the stack, it consumes this top element. If the value is Left, it executes the first provided sequence; otherwise, it executes the second sequence.
%IF-LEFT
\begin{mathpar}
  \inferrule[IF-LEFT]
  {  
  }{
    \{[(\IFLEFT\ \INSTRUCTIONONE\ \INSTRUCTIONTWO; \INSTRUCTION),
    (\StackOne, \TOR\ \TYF\ \TYS) \STACKCONCAT \STACK, \PREDICATE]\}  \cup  \SYSTEM
    \SystemTrans \\
    \{[\INSTRUCTIONONE, (\X, \TYF) \STACKCONCAT\STACK,
    \PREDICATE \wedge (\StackOne\ \EQ\ \LEFT\ \X)]\}  \cup  \{  [\INSTRUCTIONTWO, (\X, \TYS) \STACKCONCAT\STACK, \PREDICATE \wedge (\StackOne\ \EQ\ \RIGHT\ \X))]\} \cup \SYSTEM
  }
\end{mathpar}
%IF_CONS
\begin{mathpar}
  \inferrule[IF-CONS-empty]
  {  
  }{
    [(\IFCONS\ \INSTRUCTIONONE\  \INSTRUCTIONTWO; \INSTRUCTION),  (\StackOne, \TYLIST\ \TY) \STACKCONCAT\STACK, \PREDICATE] \StateTrans\  [\INSTRUCTIONTWO; \INSTRUCTION, \STACK, \PREDICATE\ \Wedge\ (\StackOne\ \EQ\ \EMPTYLIST)]}
\end{mathpar}

\begin{mathpar}
  \inferrule[IF-CONS-nonempty]
  {  
  }{
   [(\IFCONS\ \INSTRUCTIONONE\  \INSTRUCTIONTWO; \INSTRUCTION),  (\StackOne, \TYLIST\ \TY) \STACKCONCAT\STACK, \PREDICATE], \SYSTEM\ \StateTrans\  \\ [\INSTRUCTIONONE, (\HEAD, \TY) \STACKCONCAT(\{ \TAIL \}, \TYLIST\ \TY) \STACKCONCAT\STACK, \PREDICATE\ \Wedge\ ( \StackOne\ \EQ\ \{\HEAD; \STAIL \})]}
\end{mathpar}



%IF_NON
\begin{mathpar}
  \inferrule[IF-NONE-none]
  {  
  }{
    [(\IFNONE\ \INSTRUCTIONONE\  \INSTRUCTIONTWO; \INSTRUCTION),  (\StackOne, \TOPTION\ \TY) \STACKCONCAT\STACK, \PREDICATE], \SYSTEM\ \StateTrans\   [\INSTRUCTIONONE; \INSTRUCTION, \STACK, \PREDICATE\ \Wedge\ (\StackOne\ \EQ\ \NONE)]}
\end{mathpar}

\begin{mathpar}
  \inferrule[IF-NONE-some]
  {  
  }{
    [(\IFNONE\ \INSTRUCTIONONE\  \INSTRUCTIONTWO; \INSTRUCTION),   (\StackOne, \TOPTION\ \TY) \STACKCONCAT\STACK, \PREDICATE], \SYSTEM\ \StateTrans\   [\INSTRUCTIONTWO,  (\X, \TY) \STACKCONCAT\STACK, \PREDICATE\ \Wedge\ ( \StackOne\ \EQ\ \SOME\ \X)]}
\end{mathpar}

\subsubsection{Loop Instructions}
\paragraph{While loops}
%LOOP
\begin{mathpar}
  \inferrule[LOOP-true]
  {  
  }{
    [(\LOOP\ \INSTRUCTIONONE; \INSTRUCTION),  (\StackOne, \TBOOL)
    \STACKCONCAT\STACK, \PREDICATE]
    \StateTrans\
    [(\INSTRUCTIONONE; \LOOP\ \INSTRUCTIONONE; \INSTRUCTION),
    \STACK, \PREDICATE \wedge \StackOne]
  }

  \inferrule[LOOP-false]
  {  
  }{
    [(\LOOP\ \INSTRUCTIONONE; \INSTRUCTION),  (\StackOne, \TBOOL) \STACKCONCAT\
    \STACK, \PREDICATE]
    \StateTrans\
   [\INSTRUCTION, \STACK, \PREDICATE \wedge
   (\NEG\StackOne)]
   }
\end{mathpar}
\paragraph{For loops}
%%ITER
\begin{mathpar}
  \inferrule[ITER-empty]
  {  
  }{
    [(\ITER\ \INSTRUCTIONONE ; \INSTRUCTION), (\StackOne, \TYLIST\ \TY) \STACKCONCAT\STACK, \PREDICATE] \StateTrans\ [\INSTRUCTION, \STACK,  \PREDICATE\ \Wedge\  (\StackOne\ \EQ\ \EMPTYLIST)]
  }
\end{mathpar}

\begin{mathpar}
  \inferrule[ITER-nonempty]
  {  [\ITER,  (\HEAD, \TY) \STACKCONCAT\STACK, 
    Q]
    \StateTrans^*
    [ \EMPTYSTACK,  \STACK', Q']
  }{
    [(\ITER\ \INSTRUCTIONONE ; \INSTRUCTION), (\StackOne, \TYLIST\ \TY) \STACKCONCAT\STACK, \PREDICATE \wedge Q] \StateTrans \\ [(\ITER\ \INSTRUCTIONONE ; \INSTRUCTION), \{\STAIL\}\ \STACKCONCAT\STACK',  \PREDICATE \wedge Q' \Wedge  ( \StackOne\ \EQ\ \{\HEAD; \STAIL \}) ]
  }
\end{mathpar}

%\begin{mathpar}
%\inferrule[CONCAT]
 % {
 % }
 % {[(\CONCAT; \INSTRUCTION), (\StackOne, \TYLIST\ \TSTR) \STACKCONCAT\STACK,  \PREDICATE] \StateTrans 
%[\INSTRUCTION,  ("",  \TSTR) \STACKCONCAT\STACK, \PREDICATE\ \Wedge\ (\StackOne\ \EQ\ \EMPTYLIST)]}
%\end{mathpar}

%\begin{mathpar}
%\inferrule[CONCAT]
%  {
%  [\CONCAT,  (\{\TAIL\}, \TYLIST\ \TSTR) \STACKCONCAT\EMPTYSTACK, Q]
%    \StateTrans^*
%    [\EMPTYSTACK, (\StackTwo,  \TSTR) \STACKCONCAT\EMPTYSTACK, Q']
%  }
%  {[(\CONCAT; \INSTRUCTION),  (\StackOne, \TYLIST\ \TSTR) \STACKCONCAT\STACK, \PREDICATE\ \Wedge\ Q] \StateTrans \\
%[\INSTRUCTION, ( \HEAD\ \STRINGCONCAT\ \StackTwo, \TSTR) \STACKCONCAT\STACK, \PREDICATE\ \Wedge\ (\StackOne\ \EQ\ \{\HEAD; \TAIL\}) \Wedge\ Q']}
%\end{mathpar}


%MEM
\begin{mathpar}
\inferrule[MEM-empty]
  {
  }
  {[(\MEM; \INSTRUCTION), (\StackOne, \TYF) \STACKCONCAT(\StackTwo, \TYMAP\ \TYF\ \TYS) \STACKCONCAT\STACK, \PREDICATE] \StateTrans \
[\INSTRUCTION, (\FALSE, \TBOOL) \STACKCONCAT\STACK, \PREDICATE\ \Wedge\ (\StackTwo\ \EQ\ \EMPTYSET)]}
\end{mathpar}

\begin{mathpar}
\inferrule[MEM-nonempty]
  {
  [\COMPARE, (\StackOne, \TYF) \STACKCONCAT (\K, \TYF) \STACKCONCAT\EMPTYSTACK, Q]
    \StateTrans^*
    [\EMPTYSTACK,  (\VariableB, \TINT) \STACKCONCAT\EMPTYSTACK, Q']
  }
  {[(\MEM; \INSTRUCTION), (\StackOne, \TYF) \STACKCONCAT(\StackTwo, \TYMAP\ \TYF\ \TYS) \STACKCONCAT\STACK, \PREDICATE\ \Wedge\ Q] \StateTrans  \
[\MEM; \INSTRUCTION, \StackOne\ \STACKCONCAT\{\LESS\ \M\ \MORE\} \STACKCONCAT\STACK, \\ \PREDICATE\ \Wedge\ Q' \Wedge\ (\StackTwo\ \EQ\ \{\ELT\ \K\ \V; \LESS\ \M\ \MORE\})  \Wedge\ (\VariableB\ \EQ\ \ONE)]}
\end{mathpar}

\begin{mathpar}
\inferrule[MEM-nonempty]
  {
  [\COMPARE, (\StackOne, \TYF) \STACKCONCAT (\K, \TYF) \STACKCONCAT\EMPTYSTACK, Q]
    \StateTrans^*
    [\EMPTYSTACK,  (\VariableB, \TINT) \STACKCONCAT\EMPTYSTACK, Q']
  }
  {[(\MEM; \INSTRUCTION), (\StackOne, \TYF) \STACKCONCAT(\StackTwo, \TYMAP\ \TYF\ \TYS) \STACKCONCAT\STACK, \PREDICATE\ \Wedge\ Q] \StateTrans  \\
[\INSTRUCTION, (\TRUE, \TBOOL) \STACKCONCAT\STACK, \PREDICATE\ \Wedge\ Q' \Wedge\ (\StackTwo\ \EQ\ \{\ELT\ \K\ \V; \LESS\ \M\ \MORE\}) \Wedge\ (\VariableB\ \EQ\ \ZERO)]}
\end{mathpar}

\begin{mathpar}
\inferrule[MEM-nonempty]
  {
  [\COMPARE, (\StackOne, \TYF) \STACKCONCAT (\K, \TYF) \STACKCONCAT\EMPTYSTACK, Q]
    \StateTrans^*
    [\EMPTYSTACK,  (\VariableB, \TINT) \STACKCONCAT\EMPTYSTACK, Q']
  }
  {[(\MEM; \INSTRUCTION), (\StackOne, \TYF) \STACKCONCAT(\StackTwo, \TYMAP\ \TYF\ \TYS) \STACKCONCAT\STACK, \PREDICATE\ \Wedge\ Q] \StateTrans  \\
[\INSTRUCTION, (\FALSE, \TBOOL) \STACKCONCAT\STACK, \PREDICATE\ \Wedge\ Q' \Wedge\ (\StackTwo\ \EQ\ \{\ELT\ \K\ \V; \LESS\ \M\ \MORE\})  \Wedge\ (\VariableB\ \EQ\ \MINUS\ \ONE)]}
\end{mathpar}

%MAP
\begin{mathpar}
\inferrule[MAP-empty]
  {
  }
  {[(\MAP\ \INSTRUCTIONONE ; \INSTRUCTION), (\StackOne, \TYLIST\ \TY) \STACKCONCAT\STACK, \PREDICATE] \StateTrans \
[\INSTRUCTION, (\StackOne, \TYLIST\ \TY) \STACKCONCAT\STACK, \PREDICATE\ \Wedge\ (\StackOne\ \EQ\ \EMPTYLIST)]}
\end{mathpar}

\begin{mathpar}
\inferrule[MAP-nonempty]
  {
  [\INSTRUCTIONONE, (\HEAD, \TY) \STACKCONCAT\EMPTYSTACK, Q1]
    \StateTrans^*
    [\EMPTYSTACK,  (\PHEAD, \TY) \STACKCONCAT\EMPTYSTACK , Q1'] \\  [\MAP\ \INSTRUCTIONONE, (\{\TAIL\}, \TYLIST\ \TY) \STACKCONCAT\EMPTYSTACK , Q2]
    \StateTrans^*
    [\EMPTYSTACK,  (\{\PTAIL\}, \TYLIST\ \TY) \STACKCONCAT\EMPTYSTACK , Q2']
  }
  {[(\MAP\ \INSTRUCTIONONE ; \INSTRUCTION), (\StackOne, \TYLIST\ \TY) \STACKCONCAT\STACK, \PREDICATE\ \Wedge\ Q1 \Wedge\ Q2 ] \StateTrans  \\
[\INSTRUCTION, (\{\PHEAD; \PTAIL\}, \TYLIST\ \TY) \STACKCONCAT\STACK, \PREDICATE\ \Wedge\ Q1' \Wedge\ Q2'  \Wedge\ (\StackOne \EQ\ \{\HEAD; \TAIL\})]}
\end{mathpar}

\begin{mathpar}
\inferrule[UPDATE-empty-true]
  {
  }
  {[(\UPDATE; \INSTRUCTION), (\X, \TY) \STACKCONCAT(\VariableB, \TBOOL) \STACKCONCAT(\StackOne, \TYLIST\ \TY) \STACKCONCAT\STACK, \PREDICATE]\ \SystemTrans\  \\ [\INSTRUCTION, (\{\X \}, \TYLIST\ \TY) \STACKCONCAT\STACK, \PREDICATE\  \Wedge\ (\StackOne\ \EQ\ \EMPTYLIST) \Wedge\ (\VariableB\ \EQ\ \TRUE)]}
\end{mathpar}

\begin{mathpar}
\inferrule[UPDATE-empty-false]
  {
  }
  {[(\UPDATE; \INSTRUCTION), (\X, \TY) \STACKCONCAT(\VariableB, \TBOOL) \STACKCONCAT(\StackOne, \TYLIST\ \TY) \STACKCONCAT\STACK, \PREDICATE]\ \SystemTrans\  \\ [\INSTRUCTION, (\EMPTYLIST, \TYLIST\ \TY)\ \STACKCONCAT\STACK, \PREDICATE\  \Wedge\ (\StackOne\ \EQ\ \EMPTYLIST) \Wedge\ (\VariableB\ \EQ\ \FALSE)]}
\end{mathpar}

\begin{mathpar}
\inferrule[UPDATE-nonempty-true]
  {
  [\COMPARE, (\X, \TY) \STACKCONCAT (\HEAD, \TY) \STACKCONCAT\EMPTYSTACK, Q] \StateTrans^*
    [\EMPTYSTACK,  (\VariableA, \TINT) \STACKCONCAT\EMPTYSTACK, Q']
  }
  {[(\UPDATE; \INSTRUCTION), (\X, \TY) \STACKCONCAT(\VariableB, \TBOOL) \STACKCONCAT(\StackOne, \TYLIST\ \TY) \STACKCONCAT\STACK, \PREDICATE\ \Wedge\ Q] \SystemTrans
[\INSTRUCTION, (\StackOne, \TYLIST\ \TY)
\STACKCONCAT\STACK,\\ \PREDICATE\ \Wedge\ Q' \Wedge\ (\StackOne\ \EQ\ \{\HEAD; \TAIL\}) \Wedge\ (\VariableB\ \EQ\ \TRUE) \Wedge\ (\VariableA\ \EQ\ \ZERO)]}
\end{mathpar}

\begin{mathpar}
\inferrule[UPDATE-nonempty-false]
  {
  [\COMPARE, (\X, \TY) \STACKCONCAT (\HEAD, \TY) \STACKCONCAT\EMPTYSTACK, Q] \StateTrans^*
    [\EMPTYSTACK,  (\VariableA, \TINT) \STACKCONCAT\EMPTYSTACK, Q']
  }
  {[(\UPDATE; \INSTRUCTION), (\X, \TY) \STACKCONCAT(\VariableB, \TBOOL) \STACKCONCAT(\StackOne, \TYLIST\ \TY) \STACKCONCAT\STACK, \PREDICATE\ \Wedge\ Q] \SystemTrans  \\
[\INSTRUCTION, (\{\TAIL\}, \TYLIST\ \TY)
\STACKCONCAT\STACK, \\ \PREDICATE\ \Wedge\ Q' \Wedge\ (\StackOne\ \EQ\ \{\HEAD; \TAIL\}) \Wedge\ (\VariableB\ \EQ\ \FALSE) \Wedge\ (\VariableA\ \EQ\ \ZERO)]}
\end{mathpar}

\begin{mathpar}
\inferrule[UPDATE-nonempty-true]
  {
  [\COMPARE, (\X, \TY) \STACKCONCAT (\HEAD, \TY) \STACKCONCAT\EMPTYSTACK, Q] \StateTrans^*
    [\EMPTYSTACK,  (\VariableA, \TINT) \STACKCONCAT\EMPTYSTACK, Q']
  }
  {[(\UPDATE; \INSTRUCTION), (\X, \TY) \STACKCONCAT(\VariableB, \TBOOL) \STACKCONCAT(\StackOne, \TYLIST\ \TY) \STACKCONCAT\STACK, \PREDICATE\ \Wedge\ Q] \SystemTrans  \\
[\INSTRUCTION, (\{\X; \HEAD; \TAIL\}, \TYLIST\ \TY)
\STACKCONCAT\STACK, \\ \PREDICATE\ \Wedge\ Q' \Wedge\ (\StackOne\ \EQ\ \{\HEAD; \TAIL\}) \Wedge\ (\VariableB\ \EQ\ \TRUE) \Wedge\ (\VariableA\ \EQ\ \MINUS\ \ONE)]}
\end{mathpar}

\begin{mathpar}
\inferrule[UPDATE-nonempty-false]
  {
  [\COMPARE, (\X, \TY) \STACKCONCAT (\HEAD, \TY) \STACKCONCAT\EMPTYSTACK, Q] \StateTrans^*
    [\EMPTYSTACK,  (\VariableA, \TINT) \STACKCONCAT\EMPTYSTACK, Q']
  }
  {[(\UPDATE; \INSTRUCTION), (\X, \TY) \STACKCONCAT(\VariableB, \TBOOL) \STACKCONCAT(\StackOne, \TYLIST\ \TY) \STACKCONCAT\STACK, \PREDICATE\ \Wedge\ Q] \SystemTrans\  \\
[\INSTRUCTION, (\{\HEAD; \TAIL\}, \TYLIST\ \TY)
\STACKCONCAT\STACK, \\ \PREDICATE\ \Wedge\ Q' \Wedge\ (\StackOne\ \EQ\ \{\HEAD; \TAIL\}) \Wedge\ (\VariableB\ \EQ\ \FALSE) \Wedge\ (\VariableA\ \EQ\ \MINUS\ \ONE)]}
\end{mathpar}

\begin{mathpar}
\inferrule[UPDATE-nonempty-1]
  {
  [\COMPARE, (\X, \TY) \STACKCONCAT (\HEAD, \TY) \STACKCONCAT\EMPTYSTACK, Q1] \StateTrans^*
    [\EMPTYSTACK,  (\VariableA, \TINT) \STACKCONCAT\EMPTYSTACK, Q1'] \\
 [\UPDATE, (\X, \TY) \STACKCONCAT(\VariableB, \TBOOL) \STACKCONCAT( \{\TAIL\}, \TYLIST\ \TY) \STACKCONCAT\EMPTYSTACK, Q2] \StateTrans^* \\  [\EMPTYSTACK, ( \{\PTAIL\}, \TYLIST\ \TY) \STACKCONCAT\EMPTYSTACK, Q2']
  }
  {[(\UPDATE; \INSTRUCTION), (\X, \TY) \STACKCONCAT(\VariableB, \TBOOL) \STACKCONCAT(\StackOne, \TYLIST\ \TY) \STACKCONCAT\STACK, \PREDICATE\ \Wedge\ Q1 \Wedge\ Q2 ] \SystemTrans\  \\
[\INSTRUCTION; (\{\HEAD; \PTAIL\}, \TYLIST\ \TY)
\STACKCONCAT\STACK, \PREDICATE\ \Wedge\ Q1' \Wedge\ Q2' \Wedge\ (\StackOne\ \EQ\ \{\HEAD; \TAIL\}) \Wedge\ (\VariableA\ \EQ\ \ONE)]}
\end{mathpar}

\subsubsection{FAILWITH}
a Michelson program can fail, explicitly using a specific opcode

%FAILWITH
\begin{mathpar}
  \inferrule[FAILWITH]
  {
  }{[(\FAILWITH; \INSTRUCTION), (\StackOne,  \TY) \STACKCONCAT \STACK,  \PREDICATE] \StateTrans [\EMPTYSTACK, failwith (\StackOne) \STACKCONCAT\EMPTYSTACK, \PREDICATE]}
\end{mathpar}


\documentclass[a4paper,USenglish,american,cleveref, autoref, thm-restate]{oasics-v2021}
%This is a template for producing OASIcs articles. 
%See oasics-v2021-authors-guidelines.pdf for further information.
%for A4 paper format use option "a4paper", for US-letter use option "letterpaper"
%for british hyphenation rules use option "UKenglish", for american hyphenation rules use option "USenglish"
%for section-numbered lemmas etc., use "numberwithinsect"
%for enabling cleveref support, use "cleveref"
%for enabling autoref support, use "autoref"
%for anonymousing the authors (e.g. for double-blind review), add "anonymous"
%for enabling thm-restate support, use "thm-restate"
%for enabling a two-column layout for the author/affilation part (only applicable for > 6 authors), use "authorcolumns"
%for producing a PDF according the PDF/A standard, add "pdfa"

\listfiles

%\graphicspath{{./graphics/}}%helpful if your graphic files are in another directory

\bibliographystyle{plainurl}% the mandatory bibstyle

\title{Towards Contract Modules for the Tezos Blockchain} %TODO Please add

% \titlerunning{Dummy short title} %TODO optional, please use if title is longer than one line

\author{Thi Thu Ha Doan}{University of Freiburg,
  Germany}{doanha@cs.uni-freiburg.de}{https://orcid.org/0000-0002-1825-0097}{supported
by a grant from the Tezos foundation}%TODO mandatory, please use full name; only 1 author per \author macro; first two parameters are mandatory, other parameters can be empty. Please provide at least the name of the affiliation and the country. The full address is optional

\author{Peter Thiemann}{University of Freiburg, Germany}{thiemann@acm.org}{https://orcid.org/0000-0002-9000-1239}{}

\authorrunning{T.T.H Doan and P. Thiemann} %TODO mandatory. First: Use abbreviated first/middle names. Second (only in severe cases): Use first author plus 'et al.'

\Copyright{Thi Thu Ha Doan and Peter Thiemann} %TODO mandatory, please use full first names. OASIcs license is "CC-BY";  http://creativecommons.org/licenses/by/3.0/

%%TODO mandatory: Please choose ACM 2012 classifications from
%%https://dl.acm.org/ccs/ccs_flat.cfm
\begin{CCSXML}
<ccs2012>
   <concept>
       <concept_id>10011007.10011006.10011060.10011690</concept_id>
       <concept_desc>Software and its engineering~Specification languages</concept_desc>
       <concept_significance>500</concept_significance>
       </concept>
   <concept>
       <concept_id>10011007.10011006.10011041.10010943</concept_id>
       <concept_desc>Software and its engineering~Interpreters</concept_desc>
       <concept_significance>300</concept_significance>
       </concept>
 </ccs2012>
\end{CCSXML}

\ccsdesc[500]{Software and its engineering~Specification languages}
%\ccsdesc[300]{Software and its engineering~Interpreters}

\keywords{contract API, modules, static checking} %TODO mandatory; please add comma-separated list of keywords

\category{} %optional, e.g. invited paper

\relatedversion{} %optional, e.g. full version hosted on arXiv, HAL, or other respository/website
%\relatedversiondetails[linktext={opt. text shown instead of the URL}, cite=DBLP:books/mk/GrayR93]{Classification (e.g. Full Version, Extended Version, Previous Version}{URL to related version} %linktext and cite are optional

%\supplement{}%optional, e.g. related research data, source code, ... hosted on a repository like zenodo, figshare, GitHub, ...
%\supplementdetails[linktext={opt. text shown instead of the URL}, cite=DBLP:books/mk/GrayR93, subcategory={Description, Subcategory}, swhid={Software Heritage Identifier}]{General Classification (e.g. Software, Dataset, Model, ...)}{URL to related version} %linktext, cite, and subcategory are optional

%\funding{(Optional) general funding statement \dots}%optional, to capture a funding statement, which applies to all authors. Please enter author specific funding statements as fifth argument of the \author macro.

% \acknowledgements{I want to thank \dots}%optional

%\nolinenumbers %uncomment to disable line numbering

%\hideOASIcs %uncomment to remove references to OASIcs series (logo, DOI, ...), e.g. when preparing a pre-final version to be uploaded to arXiv or another public repository

%Editor-only macros:: begin (do not touch as author)%%%%%%%%%%%%%%%%%%%%%%%%%%%%%%%%%%
\EventEditors{Bruno Bernardo and Diego Marmsoler}
\EventNoEds{2}
\EventLongTitle{3rd International Workshop on Formal Methods for Blockchains (FMBC 2021)}
\EventShortTitle{FMBC 2021}
\EventAcronym{FMBC}
\EventYear{2021}
\EventDate{July 18-19, 2021}
\EventLocation{Los Angeles, California, USA (Virtual Conference)}
\EventLogo{}
\SeriesVolume{95}
\ArticleNo{4}
%%%%%%%%%%%%%%%%%%%%%%%%%%%%%%%%%%%%%%%%%%%%%%%%%%%%%%

%% structure
\newcommand{\Angle}[1]{\langle#1\rangle}

%% values
\newcommand\SUNIT{\textbf{()}}
\newcommand{\TRUE}{\textbf{True}}
\newcommand{\FALSE}{\textbf{False}}
\newcommand{\NEG}{\neg}

%% names
\newcommand{\ALS}{\textbf{als}}
\newcommand{\PAK}{\textbf{pak}}
\newcommand{\PUK}{\textbf{puk}}
\newcommand{\ADDR}{\textbf{addr}}
\newcommand{\PKH}{\textbf{pkh}}
\newcommand{\PUH}{\textbf{puh}}
\newcommand{\CODE}{\textbf{code}}
\newcommand{\BAL}{\textbf{bal}}
\newcommand{\COU}{\textbf{cou}}
\newcommand{\FAIL}{\textbf{fail}}
\newcommand{\STORAGE}{\textbf{storage}}
\newcommand{\OP}{\textbf{op}}
\newcommand{\ORIG}{\textbf{orig}}
\newcommand{\OPH}{\textbf{oph}}
\newcommand{\TIME}{\textbf{t}}
\newcommand{\CONTRACTORS}{\textbf{C}}%{\textbf{T}}
\newcommand{\PENDING}{\textbf{P}}
\newcommand{\ACCEPTED}{\textbf{A}} %obsolete
\newcommand{\MANAGERS}{\textbf{M}} 
\newcommand{\BLOCKSYSTEM}{\textbf{B}}
\newcommand{\STATUSPENDING}{\textbf{pending}} 
\newcommand{\STATUSINCLUDING}{\textbf{including}}
\newcommand{\STATUSTIMEOUT}{\textbf{timeout}}
\newcommand{\STATUSINCLUDED}{\textbf{included}}

%% types
%\newcommand{\ALS}{\textbf{als}}
\newcommand{\TPAK}{\textbf{Pak}}
\newcommand{\TPUK}{\textbf{Puk}}
\newcommand{\TADDR}{\textbf{Addr}}
%\newcommand{\PKH}{\textbf{pkh}}
\newcommand{\TPUH}{\textbf{Puh}}
\newcommand{\TCODE}{\textbf{Code}}
\newcommand{\TBAL}{\textbf{Bal}}
\newcommand{\TCOU}{\textbf{Cou}}
%\newcommand{\FAIL}{\textbf{fail}}
\newcommand{\TSTORAGE}{\textbf{Storage}}
\newcommand{\TOP}{\textbf{Op}}
%\newcommand{\ORIG}{\textbf{orig}}
\newcommand{\TOPH}{\textbf{Oph}}
\newcommand{\TTIME}{\textbf{Time}}
\newcommand{\TTEZ}{\textbf{Tez}}

%% operations
\newcommand{\TRANSFER}[5][\SUNIT]{\text{transfer $#2$ from $#3$ to $#4$ arg $#1$ fee $#5$}}
\newcommand{\ORIGINATE}[5]{\text{originate contract transferring $#1$ from $#2$ running $#3$ init $#4$ fee $#5$}}
\newcommand{\NTEZ}{\textbf{nt}}
\newcommand{\MTEZ}{\textbf{fee}}%{\textbf{m}}
\newcommand{\ID}{\textbf{id}}
\newcommand\STRING{\textbf{s}}
\newcommand\BOOLEAN{\textbf{b}}
\newcommand\INIT{\textbf{s}}
\newcommand\PARAMETER{\textbf{p}}
%% queries
\newcommand{\QRY}{\textbf{qry}}
\newcommand{\GETBALANCE}[1]{\text{balance $#1$}}
\newcommand{\GETSTATUS}[1]{\text{status $#1$}}
\newcommand{\GETSTORAGE}[1]{\text{storage $#1$}} % get contract storage
\newcommand{\GETCODE}[1]{\text{code $#1$}}
\newcommand{\GETTYPE}[1]{\text{type $#1$}}
\newcommand{\GETPUBLICKEY}[1]{\text{public key $#1$}}
\newcommand{\GETCOUNTER}[1]{\text{counter $#1$}}

\newcommand{\ACCOUNTS}{\textbf{A}}
\newcommand{\OPERATIONS}{\textbf{O}}
\newcommand{\CONTRACTS}{\textbf{S}}
\newcommand\ECS{\textbf{EC}}
\newcommand\EC[2][E]{\underline{\textbf{#1}}[#2]}
\newcommand{\EXPRS}{\textbf{E}}
\newcommand{\EXPR}{\textbf{e}}
\newcommand{\VARIABLE}{\textbf{x}}
\newcommand{\CONSTANT}{\textbf{c}}
\newcommand{\CONS}{\textbf{cons}}
\newcommand{\NIL}{\textbf{nil}}
\newcommand\LEFT{\textbf{left}}
\newcommand\RIGHT{\textbf{right}}
\newcommand{\AND}{\textbf{and}}
\newcommand{\OR}{\textbf{or}}
\newcommand{\NOT}{\textbf{not}}
\newcommand{\PLUS}{\textbf{+}}
\newcommand{\MINUS}{\textbf{-}}
\newcommand{\EQUAL}{\textbf{=}}
\newcommand{\LESS}{\textbf{$<$}}
\newcommand{\NI}{\textbf{$n_i$}}
\newcommand{\EI}{\textbf{$e_i$}}
\newcommand\FST{{\bf fst}}
\newcommand\SND{{\bf snd}}
\newcommand{\RETURN}{\textbf{r}}
\newcommand\RAISE{\textbf{raise}}
\newcommand\TRY{\textbf{try}}
\newcommand\EXCEPT{\textbf{except}}
\newcommand\MATCH{\textbf{match}}
\newcommand\WITH{\textbf{with}}
\newcommand\PATTERN{\textbf{pat}}
\newcommand\SOME{\textbf{some}}
\newcommand\NONE{\textbf{none}}
\newcommand\VAL{\textbf{v}}
\newcommand\SC{\textbf{sc}}

\newcommand\TYPE{\textbf{ty}}
\newcommand\TINT{\textbf{Int}}
\newcommand\TUNIT{\textbf{Unit}}
\newcommand\TBOOL{\textbf{Bool}}
\newcommand\TSTRING{\textbf{String}}
\newcommand\TSTATUS{\textbf{Status}}
\newcommand\TPAIR{\textbf{Pair}}
\newcommand\TLIST{\textbf{List}}
\newcommand\TSUM{\textbf{Or}}
\newcommand\TOPTION{\textbf{Option}}
\newcommand\TEXCEPTION{\textbf{Exception}}

\newcommand\QOP{\textbf{qop}}
\newcommand\INT{\textbf{i}}

\newcommand\TEnv{\Gamma}
\newcommand\JTypeExpr[3]{#1 \vdash #2 : #3}

\newcommand{\NODE}{\textbf{N}}
\newcommand{\BLOCKCHAIN}{\textbf{B}}

%% functions
\newcommand{\CHECKACC}{\textup{chkAcc}}
\newcommand{\CHECKARG}{\textup{chkArg}}
\newcommand{\CHECKGAS}{\textup{chkFee}}
\newcommand{\CHECKID}{\textup{chkId}}
\newcommand{\CHECKBAL}{\textup{chkBal}}
\newcommand{\CHECKCOU}{\textup{chkCount}}
\newcommand{\CHECKPUB}{\textup{chkPub}}
\newcommand{\CHECKPUH}{\textup{chkPuh}}
\newcommand{\CHECKPRG}{\textup{chkPrg}}
\newcommand{\CHECKEXIST}{\textup{chkEx}}
\newcommand{\CHECKINIT}{\textup{chkInit}}
\newcommand{\CHECKPARA}{\textup{chkInp}}
\newcommand{\UPDATECOU}{\textup{updCount}}
\newcommand{\UPDATECONSTR}{\textup{updConstr}}
\newcommand{\UPDATESUCC}{\textup{updSucc}}
\newcommand{\UPDATESTORAGE}{\textup{updStor}}
\newcommand{\GENERATEOPH}{\textup{genOpHash}}
\newcommand\GENERATEHASH{\textup{genHash}}

%% transition relations
\newcommand{\ExprTrans}[1][{}]{\stackrel{#1}\longrightarrow_E}
\newcommand{\BlockTrans}[1][{}]{\stackrel{#1}\longrightarrow_B}
\newcommand{\NodeTrans}[1][{}]{\stackrel{#1}\longrightarrow_N}
\newcommand{\SystemTrans}{\longrightarrow}

%% typing related
\newcommand{\EmptyEnv}{\cdot}

%% evaluation contexts
%\newcommand\EC[1]{\epsilon[#1]}

%% metavariables
\newcommand\STATUS{\textbf{st}}
\newcommand\ERROR{\textbf{err}}
\newcommand\ERRPRG{\textbf{excPrg}}
\newcommand\ERRBAL{\textbf{excBal}}
\newcommand\ERRCOUNT{\textbf{excCount}}
\newcommand\ERRFEE{\textbf{excFee}}
\newcommand\ERRPUK{\textbf{excPub}}
\newcommand\ERRPUH{\textbf{excPuh}}
\newcommand\ERRARG{\textbf{excArg}}
\newcommand\ERRINIT{\textbf{excInit}}

\newcommand\partialto\hookrightarrow
\newcommand\DOM{\textit{dom}}
\newcommand\Nat{\textbf{N}}

%%% Local Variables:
%%% mode: latex
%%% TeX-master: "model"
%%% End:


\begin{document}

\maketitle

%TODO mandatory: add short abstract of the document
\begin{abstract}
  Programmatic interaction with a blockchain is often clumsy.
  Many interfaces handle only loosely structured data, often in JSON
  format, that is inconvenient to handle and offers few guarantees.

  Contract modules provide a statically checked interface to interact
  with contracts on the Tezos blockchain. A module specification
  provides all types as well as information about potential failure
  conditions of the contract. The specification is checked against the
  contract implementation using symbolic execution. An OCaml module is
  generated that contains a function for each entry point of the
  contract. The types of these functions fully describe the interface
  including the failure conditions and guarantee type-safe and
  sometimes fail-safe invocation of the contract on the blockchain.
\end{abstract}

\section{Introduction}
\label{sec:introduction}

Contracts on the blockchain rarely run in isolation. To be useful
beyond shuffling tokens between user accounts, they need to interact
with the outside world. On the other hand, the outside world also
needs to interact by initiating transactions and starting contracts
that feed information into the blockchain. One direction is addressed
by oracles that watch certain events on the blockchain, create a
response by calculation or gathering data, and then invoke a callback
contract to inject this response into the chain. Trust is an essential
aspect for an oracle.

The other direction is about automatizing certain processes in
connection with the blockchain. For example, opening or closing an
auction according to a schedule, programming a strategy for an
auction, or creating an NFT. To this end, an interface is needed to
invoke contracts safely. Existing interfaces are lacking because they
are essentially untyped (string-based or JSON-based) and often low
level because they require dealing directly with RPC interfaces. Trust
is not needed because the process runs on behalf of a certain user.

We propose contract modules that provide a clean, language-integrated
way to interact with a blockchain from a host language (OCaml in our
case). They abstract over underlying string-based interfaces and
details like fee handling. They provide a high-level typed interface
which reduces a contract invocation to a function call in the host
language.

The contract modules approach does not provide a fixed API, but
rather generates a specific interface for each contract with one
function for each entry point of the contract. This
interface is statically checked against the contract implementation to
ensure type safety and exception safety. That is, values passed to an
interface function do not lead to type mismatches when invoking the
underlying contract. Morever, every failure condition arising during
contract execution is handled by proper error reporting according to
the interface.

Our work is situated in the context of the Tezos blockchain, which
supports Michelson as its low-level contract language, and the
host language OCaml, which comes with an expressive polymorphic type system
as well as a powerful module system that we
enhance with contract modules. 

\section{Context}
\label{sec:context}

Tezos is a third generation, account-based, self amendable
blockchain \cite{tezos-whitepaper}. It employs a proof-of-stake consensus protocol, which
includes ways to evolve the protocol itself. The consensus protocol is
executed by so-called bakers and their proposed blocks are checked by
validators. They receive some compensation in the form of
tokens (Tezzies) for their work. According to proof-of-stake,
bakers and validators are nodes elected by the Tezos network
according to their token balance. 

Each Tezos contract is associated with an account as well as some
storage. Contracts are pure functions of type $\mathit{parameter} \times
\mathit{storage} \to \mathit{operation\ list} \times
\mathit{storage}$, where the types \textit{parameter} and
\textit{storage} are depend on the specific contract while the type
\textit{operation} is fixed by the Tezos system. When a contract is invoked with a parameter,
the blockchain provides the current storage and updates it with the
second, storage component of its return value. The first component of
the return value is a list of blockchain operations (contract deployments, token
transfers, contract invocations, and delegation of baking rights) that
are executed transactionally after the first invocation terminates. Each invocation may be
accompanied with an amount of tokens that are added to the current
account balance of the callee contract.

Contracts are implemented in the language Michelson, a statically typed
stack-based language. Each contract has fixed types for its parameter
and for its storage. The storage is initialized when the contract is
deployed. Besides primitive types like unit, int, bool, address, and string,
there are pairs, sums, functions, lists, and maps along with a range
of domain-specific types (operation, key, signature, timestamp,
key\_hash, contract, mutez --- for tokens, and so on) most of which
can serve as types for storage and parameters.

A Michelson contract has a single default entry point. However, the \textit{parameter} type is
typically a sum type and each component of the sum can serve as a subsidiary entry point. 

\section{An auction contract}
\label{sec:an-auction-contract}
\begin{lstlisting}[float,caption={Simple auction contract (auction.tz)},captionpos=b,label={lst:auction-header},language=michelson,numbers=none,emph={close,bid,bidding,owner,hi_bidder},emphstyle=\underbar]
parameter (or (unit %close) (unit %bid));
storage (pair (bool %bidding)
              (pair (address %owner)
                    (address %hi_bidder)));
\end{lstlisting}

As a concrete example, we consider a simple auction contract with the
header shown in Listing~\ref{lst:auction-header}. 
This contract has two entry points, \lstinline/close/ and \lstinline/bid/,
expressed by giving the single parameter a sum type. To call the
entry point \lstinline/close/ we invoke the contract with parameter \lstinline/Left ()/
otherwise we use \lstinline/Right ()/, where \lstinline/()/ is the sole
value of type \lstinline/unit/. The contract's storage is a nested pair which
contains a boolean flag and two addresses.

The contract works as follows. It is deployed with storage
\lstinline/(true,(owner,owner))/ which indicates that bidding is
allowed and the contract owner is currently the highest
bidder. On deployment the owner deposits an initial balance to indicate the
minimum bid. Closing the contract transfers the balance to the
owner. Closing is restricted to the owner. Closing as well as bidding fails
if the auction is
closed. If bidding is open and the amount of tokens accompanying the bid exceeds the
current highest bid, the current bidder replaces the previous highest
bidder and the previous highest bidder is reimbursed. Otherwise,
bidding fails.

To invoke this contract from an OCaml program, we generate an
OCaml module, say \lstinline/Auction/, from a specification of the
contract. This module contains two functions
\lstinline/close/ and \lstinline/bid/ corresponding to the
entry points. The type of these entry points reflects further properties
of these entry points as well as the ways in which an entry point might
fail.

Besides the obvious, technology induced ways that a contract invocation might fail
(insufficient gas price offered, insufficient gas to complete, timeout
due to lack of connectivity, etc) a Michelson contract can fail due to
a programmer induced condition caused by the
instruction \lstinline/FAILWITH/. It terminates contract execution
with an error message which is reported back to the caller. This error
message includes the top value on the stack.

We consider the technological failures like Java's unchecked
exceptions, but we wish to deal with the explicit failures like
checked exceptions~\cite{DBLP:conf/oopsla/AnconaLZ01}.
Our generated code handles failures in a suitable
error monad that makes the failures explicit in a custom
datatype.\footnote{Alternatively, errors could be modeled using OCaml
  exceptions, but we choose to stay within the monadic framework that
  is used by existing Tezos APIs.}

\begin{lstlisting}[float,captionpos=b,caption={Example contract module},label={lst:contract-module-example}]
contract type Auction = sig
  paid entrypoint bid () 
  raises "closed"               (** auction closed *)
       | "too low"              (** bid too low    *)

  entrypoint close ()
  raises "closed"               (** auction closed *)
       | "not owner"            (** caller cannot close *)
end
\end{lstlisting}
Listing~\ref{lst:contract-module-example} shows a contract module for
the auction contract.  It declares the entry point \lstinline/bid/ as
\lstinline/paid/, i.e., it should be invoked with a
non-zero amount of tokens, it states the pattern \lstinline/()/ for the
input value of type \lstinline/unit/, and it specifies two 
failure messages that we wish to deal with programmatically.
The \lstinline/close/ entry point is similar, but it is not
\lstinline/pair/. This contract reflects the understanding of the
programmer with the intention that the \lstinline/raises/ clauses
cover all failures that can arise during execution of the respective
entry point. 

% module type MyContract = sig 
\begin{lstlisting}[float,captionpos=b,caption={Generated signature},label={lst:generated-signature}]
  type bid_errors = 
     | bid_closed      (** auction closed*)
     | bid_too_low     (** bid received is too low *)

  val bid
    : Tezos.pukh -> ?fee:Tezos.mutez -> amount:Tezos.mutez
    -> (Tezos.status, bid_errors) Tezos.monad

  type close_errors = 
     | close_closed    (** auction closed *)
     | close_not_owner (** caller cannot close the auction *)

  val close
    : Tezos.pukh -> ?fee:Tezos.mutez
    -> (Tezos.status, close_errors) Tezos.monad
\end{lstlisting}
% end 
Listing~\ref{lst:generated-signature} contains an OCaml module
signature as it would be generated from the contract module. The
module \lstinline/Tezos/ supposedly contains types and other low-level
Tezos-specific definitions. The type \lstinline/pukh/ for public
key hashes identifies contracts, the type \lstinline/mutez/ stands for
Tezos tokens, the type \lstinline/status/ reflects the internal return
status, and \lstinline/monad/ is an internal monad type. The signature
declares a function and an error type for each entry point.

The error types mirror the raises clauses. The first argument of each
function is the address of the contract, then an optional argument for
the transaction fee, an argument for passing an amount of tokens
(only for a paid entry point), the next argument would be for
the parameter; it is omitted here because its type is
\lstinline/unit/. The return type refers to the specific error type.

\section{Simple Checking}
\label{sec:check-contr-against}

We check the contract by symbolic execution against its
specification in the contract module. Symbolic execution proceeds by calculating a symbolic stack at
each transition from 
one Michelson instruction to the next. The symbolic interpreter is fully typed and rejects ill-typed
Michelson programs, if the ill-typed part is reachable from the default entry point. The initial
stack is calculated 
from the storage type, the parameter type, and the entry point.
As we aim to keep symbolic
values as concrete as possible, some instructions may turn out to be
unreachable or only reachable under certain conditions. This way, we
obtain, for each entry point, a set of final symbolic stacks along with
a path condition indicating when this final state is
reachable. Moreover, for each \lstinline/FAILWITH/ instruction we
obtain a symbolic value for the reported message and a path condition
indicating reachability of this instruction. We employ the SMT solver
Z3 \cite{DBLP:conf/tacas/MouraB08,DBLP:conf/setss/BjornerMNW18} to
check the feasibility of a path condition. 

Here are some simple examples of checkable properties.

For each entry point, we collect the set of reachable instructions.
For example, the  \lstinline/AMOUNT/ instruction obtains the amount of
tokens sent with a contract invocation.
It should not be possible to reach that instruction from an unpaid
entry point  like \lstinline/close/. This property is straightforward
to check from the path condition generated for the \lstinline/AMOUNT/
instruction. 

For each entry point, we collect the set of reachable \lstinline/FAILWITH/ instructions along with
their path condition and their arguments. In most cases the symbolic interpreter finds a concrete
argument for each \lstinline/FAILWITH/ instruction because the typical usage pattern is to push a
concrete (string) value on the stack immediately preceding the \lstinline/FAILWITH/.
It remains to check that each argument to \lstinline/FAILWITH/ should be accounted for by one
\lstinline/raises/ clause.

Checking the simple auction contract in this way already flags an
omission in the contract module
(Listing~\ref{lst:contract-module-example}). The problem is that
bidding returns the previous highest bid as part of the transaction
where the highest bidder is stored as a value of type
\lstinline/address/.  However, to receive a token transfer, this
address must by cast (by the Michelson implementation of the contract)
to an implicit contract of Michelson type \lstinline/(contract unit)/. This cast is unavoidable as a
value of contract type cannot be stored. However, the cast may fail and its failure leads to an
error condition that is reported with a \lstinline/FAILWITH/ instruction that is not covered by the
contract module.\footnote{We generated the Michelson code for this example using the Liquidity
  compiler (\url{https://www.liquidity-lang.org/}). While the failures for \lstinline/closed/ and
  \lstinline/too low/ are explicit in the source program, the compiler inserts the casts into the contract
  type automatically using the error message \texttt{No entrypoint default with parameter type
    unit}.}  The entry point for closing has the same issue with the address of the owner, who is
scheduled to receive the balance of the contract.

The best way to address this issue would be to assert that the addresses stored for the owner and
the highest bidder always cast successly into the \lstinline/(contract unit)/ type so that the
stated module type can be retained. Unfortunately, doing so requires control over the initialization
of the storage, which is part of the deployment of the contract. At present, our design for contract
modules does not include support for the deployment of a contract. Moreover, it is not clear whether
it makes sense to support it as there is no guarantee
that a contract will be deployed using the API generated from a contract module.

The quick fix is to add \lstinline/raises/ clauses for the \texttt{No entrypoing \dots}
message. With this fix, we successfully check the module description against the Michelson
implementation of its contract. Another fix would be to rewrite the contract to enable the
dynamic creation of auctions.

\section{Advanced Checking}
\label{sec:advanced-checking}
\begin{lstlisting}[float,captionpos=b,caption={Enhanced contract module},label={lst:safer-contract-module},numbers=left]
contract type SaferAuction = sig
  storage (Pair (bidding : bool) 
                (Pair (owner : address) (hi_bidder : address)))

  entrypoint close ()
  requires bidding raises "closed"        (** auction closed*)
  requires (SOURCE = owner) raises "not owner"
  ensures not bidding
  ensures (post.BALANCE = 0)
  ensures (TRANSFER_TOKENS unit BALANCE hi_bidder)

  paid entrypoint bid () 
  requires bidding raises "closed"        (** auction closed*)
  requires (AMOUNT > pre.BALANCE) raises "too low" (** low bid *)
  ensures bidding 
  ensures (post.BALANCE = AMOUNT)
  ensures (post.hi_bidder = SOURCE)
  ensures (TRANSFER_TOKENS unit pre.BALANCE hi_bidder)

  invariant (post.owner = owner)
  invariant (post.bidding => bidding)
  invariant (post.hi_bidder = hi_bidder or post.hi_bidder = SOURCE)
end
\end{lstlisting}

As it is expensive to invoke a contract just to find out that it
fails, we propose to extend entry point specifications with
preconditions along with some global invariants as shown in Listing~\ref{lst:safer-contract-module}. The
idea is that the generated OCaml module tries to check the
preconditions off-chain before invoking the contract. To this end, the
off-chain code needs to obtain properties like balance, storage etc of the
contract, but this information is available from a Tezos node without a fee! Of course, such an
off-chain check is prone to race conditions as concurrent contract invocations from other parties
may interfere and change the data before our module gets a chance to invoke the contract.

Generally, a specification can refer to the values available to the contract. For example, the
instructions \lstinline/SOURCE/\footnote{The execution of a Michelson contract is part of
  a transaction, which can encompass several contract executions. The SOURCE of a Michelson contract
  is the originator of the entire transaction.}, \lstinline/BALANCE/, and \lstinline/AMOUNT/ refer
to the respective values. As the current \lstinline/BALANCE/ includes the \lstinline/AMOUNT/ sent
with the transaction, we write \lstinline/pre.BALANCE/ ($=$ \lstinline/BALANCE/ $-$
\lstinline/AMOUNT/) for the balance before the transaction starts and \lstinline/post.BALANCE/ for
the balance after the transaction finishes. The latter is calculated by subtracting the amounts
transferred from the current \lstinline/BALANCE/. The existence of a token transfer in the returned
operation list is indicated by
the respective \lstinline/TRANSFER_TOKENS/ instruction.  The components of the storage are referred
to by name. We distinguish the outgoing storage by prepending \lstinline/post./ as in
\lstinline/post.owner/.

For each clause \lstinline/requires/ $\Phi$ \lstinline/raises/ $s$, we take the path condition
$\Theta$ for a \lstinline/FAILWITH/ instruction with argument matching the string $s$ and check that
$\Phi \wedge \Theta$ is not satisfiable in the context given by the initial stack for the entry point.

From all \lstinline/requires/ $\Phi_i$ and \lstinline/ensures/ $\Psi_j$ clauses, we check that $\neg
(\bigwedge\Phi_i \to \bigwedge \Psi_j)$ is unsatisfiable in the context given by the initial stack
for the entry point and its corresponding final stack and path condition.

We discuss two of the preconditions to highlight the properties that 
need to be analyzed and where race conditions may interfere. 

The precondition \lstinline/SOURCE = owner/ of \lstinline/close/ can
be checked off-chain because the owner's address is part of the
storage. However, it is in general unsound to perform such a test
off-chain because the owner's address could change if an entry point
changes that component of the storage. To safely check this
precondition, all other entry points must preserve the \lstinline/owner/'s address, which is indeed
the case by the postcondition in line~17. This postcondition is verified as outlined above.

% Moreover, the gathering of instructions must build path predicates,
% such that each \lstinline/FAILWITH/ instruction comes with a predicate
% that must be true to reach the instruction. In contract
% implementation, the path predicate is \lstinline/sender != owner/. As
% the conjunction of path predicate and precondition is unsatisfiable
% and because the \lstinline/owner/ component is constant, an off-chain
% test for \lstinline/sender = owner/ precisely predicts whether the
% failure condition arises.

The situation is slightly more complex at the \lstinline/bid/
entry point. The failure \lstinline/"closed"/ is guarded by
\lstinline/bidding/. As the \lstinline/bidding/ component of the state can change,
a precise prediction is not possible. A closer look reveals some
subtlety. If \lstinline/bidding/ is true, then the flag may have
changed by some interleaved call to \lstinline/close/. However, if
\lstinline/bidding/ is false, then there is no point in invoking the
contract because \lstinline/bidding/ will never be reset to true.
We address this situation by verifying the global invariant
\lstinline/post.bidding => bidding/. 

For the failure \lstinline/"to low"/, the analysis is very similar:
we need to know that there is no successful execution of
\lstinline/bid/ after an execution of \lstinline/close/. Moreover,
each invocation of \lstinline/bid/ raises the balance of the contract
monotonically. Thus, if the off-chain check
\lstinline/AMOUNT > pre.BALANCE/ fails, we can be sure that the contract
invocation will also fail; either because someone closed the auction or
because the balance is at least as high as in the off-chain sample. Checking \lstinline/pre.BALANCE/
off-chain is particularly simple, because it is the current balance of the account.

\section{Symbolic interpretation of Michelson}
\label{sec:symb-interpr-mich}

Michelsym, our symbolic interpreter for Michelson, works in two stages. In the first stage, it
calculates symbolic stacks between each pair of reachable instructions. The underlying symbolic
domain comprises all concrete values, supports the type system, and generates a term representation
for symbolic values. Michelsym collects a path predicate that is extended at each conditional, but
which remains uninterpreted. If symbolic execution reaches certain instructions (most notably
\lstinline/FAILWITH/), Michelsym records the argument value and the path condition. 

Presently, Michelsym works on Michelson files which result from compiling the examples
provided with the Liquidity compiler. It generates human-readable output as well as output in the
SMTlib format suitable for SMT-solvers like Z3. This output needs to be weaved together manually
with the formulas generated from a contract module.

We plan to revise Michelsym so that it directly communicates with Z3 to directly check for
unsatisfiable path conditions and to be better integrated with the contract module frontend. 

\section{Related work}
\label{sec:related-work}
%RPC APIs, often provided by current blockchains such as Ethereum \cite{eth-whitepaper} and Tezos \cite{tezos-whitepaper}, use loosely structured data, such as a JSON-based format, which is difficult for a programmatic program to handle. As a result, there is increasing work on providing better programmatic interfaces to the blockchains. Web3.js \cite{web3.js} provides an Ethereum JavaScript API and offers Java Script users a convenient interface to interact with the Etherum blockchain. Later, Web3.py \cite{web3.py}, derived from Web3.js, is developed to provide a Python library for interacting with Ethereum. Tezos prove a PRC API that utilizes JSON format for both input and output data, which is neither convenient nor programmatic.



Smart contract-based applications often require interaction between a smart contract on the blockchain and the outside world. However, smart contracts cannot connect to external sources on their own. This is where oracles \cite{oracle-patterns,call-action-oracle} come into play. Oracles act as a bridge between smart contracts and external sources. Namely, they collect and verify external information and make it available to smart contracts on the blockchain. Several research works have been conducted to provide oracle solutions for the Blockchain. Adler et al.\cite{blockchain-oracles} proposed a framework to provide developers with a guide for incorporating oracles into blockchain-based applications. Oracles may need to observe the state of the chain to determine what information to send. In addition, oracles transmit data from external sources to the blockchain. Therefore, they would need to have a programmatic interface to interact with the blockchain. %Practically, Ellis et al. \cite{chainlink-whitepaper} proposed the decentralized oracle network Chainlink to provide reliable, tamper-proof input for smart contracts on any blockchain. ChainLink securely provides data to APIs on behalf of a smart contract.

%An oracle responds to an event by  transmitting information to a smart contract through a smart contract invocation. In addition. the original smart contract must inherit an additional smart contract from Oracle in order to interact with external data sources. Deployment costs are increased when using Ethereum smart contracts with Oracle as the data carrier. To deal with this problem, the paper \cite{LIU2019590} proposes a low-cost data carrier architecture for a blockchain-enabled IoT environment.

The basic idea of our advanced checking, namely precondition checking,
is inspired by JML, the Java modeling
language~\cite{Leavens2006DesignBC, 101007}, in which the behavior of 
program components is described as a contract between Java
program and its clients. This contract specifies
preconditions that must be satisfied by clients and postconditions
that are guaranteed by the program. A precondition supplied with a
client call must be verified before a function defined by the program
is called, and the program guarantees that the postconditions are
satisfied in return after the call. The original idea of using
preconditions and postconditions dates back to Hoare's paper
\cite{101145}. Software contracts have also been proposed for
blockchain \cite{DBLP:journals/fbloc/Bartoletti20}.
% The executable contracts are written as program code in
% the programming language.
In our approach, the safe contract module in the OCaml language comes
close to contracts in this sense. Several
applications are based on JML \cite{Tran2017}. Ahrendt et
al. \cite{keY} propose the KeY framework for deductive software
verification.  

Our contract module specifies preconditions and then off-chain checks whether a user call satisfies those preconditions. Symbolic execution plays an important role in the preconditions checking in our method. A smart contract is verified against its specification in the contract module by symbolic execution. In a paper on symbolic execution \cite{Hentschel}, Hentschel et al. proposed the symbolic execution debugger (SED) platform, which is based on the KeY framework. The platform SED has a static symbolic execution engine for sequential programs.

\section{Conclusion}
\label{sec:conclusion}

Current blockchains often provide low-level interfaces to interact
with smart contracts. These interfaces work with loosely structured
without static guarantees. This paper presents ongoing research on the
programmatic interaction with smart contracts on the Tezos blockchain
that could benefit developers of mixed applications and
oracles comprised of on-chain and off-chain parts. The approach does
not provide a general API, but targets each 
individual smart contract by generating a specialized contract module
that provides a typed high-level interface from a contract
specification. In doing so, errors from contract calls are explicitly
specified in a user-defined data type. A contract call is wrapped in a
fully typed and integrated OCaml function. In addition, the wrapper can check preconditions before the 
actual call to reduce the waste of gas of a failed call.


While our conceptual approch is applicable and would be beneficial for other blockchains, the actual
implementation is very much tied to the Tezos blockchain. The key asset here is the symbolic
interpreter which is hardcoded for Michelson and adapted to the peculiarities of the Tezos
blockchain. By targeting Michelson, our work is applicable to all languages running on Tezos, but
a similar tool would have to be developed from scratch for another blockchain.

%%
%% Bibliography
%%

%% Please use bibtex, 

\bibliography{biblio}

\end{document}

\documentclass[a4paper]{llncs}
\usepackage[english]{babel}
\usepackage[utf8]{inputenc}
\usepackage{amsmath}
\usepackage{amssymb}
\usepackage{graphicx}
\usepackage[colorinlistoftodos]{todonotes}
\usepackage{mathpartir}
\usepackage{fixme}
\usepackage{listings}

\lstset{
  language=caml,
  morekeywords={requires,sender,amount,balance,storage,contract,entrypoint,module,hash,paid,raises,sig,val},
  % predefined types
  classoffset=1,
  morekeywords={nat,int,unit,string,pukh,status,tz},
  keywordstyle=\color{blue},
  classoffset=2,
  morekeywords={Pair,Left,Right},
  keywordstyle=\color{magenta},
  backgroundcolor=\color{brown!20},
  captionpos=b,
  literate={->}{{$\to$}}1
}

\lstdefinelanguage{michelson}{
  % basicstyle=\ttfamily{},
  classoffset=0,
  morekeywords={parameter,storage,code},
  keywordstyle=\bfseries, 
  % predefined types
  classoffset=1,
  
  morekeywords={nat,int,unit,string,or,pair,option,address},
  keywordstyle=\color{blue},
  classoffset=2,
  morekeywords={CAR,CDR,CONS,DUP,EQ,FAILWITH,IF,IF_LEFT,LE,LEFT,NIL,DROP,PUSH,RIGHT,SENDER,SUB},
  keywordstyle=\color{magenta},
  backgroundcolor=\color{pink!20}
}

\title{Whitepaper: Contract Modules}
\author{Peter Thiemann}
\institute{}

\date{\today}

\begin{document}
\maketitle
\pagestyle{plain}

\section{A simple example}
\label{sec:motivation}


Consider this Michelson code fragment for implementing an auction contract:
\begin{lstlisting}[language=michelson,caption={Michelson code example},label={lst:code-example}]
parameter (or (unit %close)
              (unit %bid));
storage (pair bool (pair address address));
code { DUP; CAR; 
       IF_LEFT {
       # %close
         DROP; CDR; CAR; SENDER; EQ;
         IF {} { PUSH string "not owner"; FAILWITH; }
         # ...
       }
       {
       # %bid
        DROP; AMOUNT; PUSH mutez 10000; COMPARE; LE;
        IF { FAILWITH; } {} # fails with argument value
        # ...
        { PUSH string "closed"; FAILWITH; }
        # ...
        { PUSH string "bid too low"; FAILWITH; }
        # ...
        { PUSH int 42; FAILWITH; }
        # ...
       }
     }
\end{lstlisting}

Here is a simple \textbf{contract module} that describes / specifies
the above Michelson code. 
\begin{lstlisting}[caption={Contract module example},label={lst:contract-module-example}]
contract type MyContract = sig
  paid entrypoint %bid () 
  raises "closed"                 (** auction closed*)
       | "bid too low" as too_low (** bid received is too low *)
       | 42 as the_answer         (** failwith received argument 42 *)
       | (_ : int) as failedwith  (** failwith received an integer arg *)

  entrypoint %close ()
  raises "not owner"         (** caller cannot close the auction *)
end
\end{lstlisting}

To specify the contract module, we store
the body of the \lstinline/sig ... end/ in a
\texttt{.tzm} file where the basename matches the one of the
\texttt{.tz} file which implements the contract.

A contract module gets compiled to an \texttt{.mli} file which contains a
valid OCaml signature and the corresponding \texttt{.ml} file
implementing the signature. 

Each \lstinline/entrypoint/ is mapped to a function. The entrypoint's
parameter is described by a pattern.  Variables (OCaml identifier) or
wildcards \lstinline/_/ in the pattern will be used as formal
parameters in generating the implementation. For \lstinline/_/ a
standard name will be generated. Each variable or wildcard needs a
type annotation. Constructors in the pattern will be
omitted from the interface, but automatically inserted in the contract
invocation. 

A \lstinline/paid/ entrypoint requires the caller to specify an amount of tokens to
accompany the contract invocation. Accordingly, the generated function
has a mandatory argument to specify this amount. Otherwise, the entrypoint is unpaid. 

The \lstinline/raises/ clause specifies values that the contract might
pass to the \texttt{FAILWITH} instruction. The translation generates
an error type from the clause that describes all possible
failures. The \lstinline/as/ clause 
maps the matched value to an OCaml identifier, which is used to
generate the corresponding case in the error type. There is a standard
translation (to be specified) that maps a value without an
\lstinline/as/ clause to an identifier.

The \lstinline/raises/ clause must be exhaustive. Any failure arising
during execution of the contract must be captured by one of the
specified cases. If in doubt, include a catchall default case.

Alternatively and equivalently, contracts can be specified using
patterns on the default entrypoint.

\begin{lstlisting}[caption={Alternative syntax for entrypoints},label={lst:alternative-syntax}]
  paid entrypoint bid (Right ())
...
  entrypoint close (Left ())
\end{lstlisting}

The specification does not have to be exhaustive. The contract may
have further entrypoints that are not described by the module.

\clearpage{}
\section{A safer example}
\label{sec:safe-example}

Alternatively, one may specify preconditions that are checked before
the contract is deployed to the blockchain. 

\begin{lstlisting}[caption={Safer contract module},label={lst:safer-contract-module}]
contract type SaferContract = sig
  storage (Pair (bidding : bool) 
                (Pair (owner : address) 
                      (highest_bidder : address)))

  entrypoint %close ()
  requires (sender = owner) raises "not owner"

  paid entrypoint %bid () 
  requires bidding raises "closed" (** auction closed*)
  requires (amount > balance)
  requires (amount > 10000)
  raises "bid too low" as too_low (** bid received is too low *)
       | 42 as the_answer         (** failwith received argument 42 *)
end
\end{lstlisting}
The \lstinline/storage/ clause defines a typed pattern to match the
current storage of the contract. Moreover, it generates a suitably
typed getter function (also named \lstinline/storage/) 
to access the current storage from OCaml. 

The \lstinline/%close/ entrypoint comes with a \lstinline/requires/
clause that specifies a precondition for invoking the contract. This
precondition is checked in the OCaml interface \emph{before} invoking
the contract. The entrypoint does not have a \lstinline/raises/
clause anymore, which means that this precondition implies that no
failures can occur.

\textbf{For this reasoning to be true, we need to know that the owner
  component of the storage is constant through all contract executions.}

For the \lstinline/%bid/ entrypoint, the \lstinline/bidding/ component
must be true. However, we cannot guarantee that there is nobody
closing the contract between accessing the storage and the
actual contract invocvation. Hence, the corresponding
\lstinline/raises/ clause must remain in place.

\textbf{To safely reject when \lstinline/bidding/ is false, we need to know
  that there is no contract execution that can make it true, again.}

Analogous reasoning applies to the condition
\lstinline/amount > balance/. Here, we also need to know that the
balance cannot decrease, which is non-trivial. However, we can accurately check that
\lstinline/amount > 10000/ before the contract invocation, which rules
out the unspecific integer failure.

The interface of the generated functions remains the same, but some of
the errors are generated without involving the blockchain.

\section{Generated code}
\label{sec:generated-code}


The contract module is compiled to an OCaml module which provides a typed API to the
contract. It includes the documentation comments from the contract
module. Internally, the typed API builds on the untyped API developed
by us. 

\begin{lstlisting}[caption={Generated signature},label={lst:generated-signature}]
module type MyContract =
sig
  type bid_errors = 
     | closed       (** auction closed*)
     | too_low      (** bid received is too low *)
     | the_answer   (** failwith received argument 42 *)
     | failedwith of int (** failwith received an integer arg *)

  val bid
    : Tezos.pukh -> ?fee:Tezos.tz -> amount:Tezos.tz
    -> (Tezos.status, bid_errors) Tezos.monad

  type close_errors = 
     | not_owner    (** caller cannot close the auction *)

  val close
    : Tezos.pukh -> ?fee:Tezos.tz -> ?amount:Tezos.tz
    -> (Tezos.status, close_errors) Tezos.monad
end
\end{lstlisting}

\begin{itemize}
\item The first argument is the public key hash of the caller. 
\item The \lstinline/paid/ entrypoint requires an \lstinline/amount/
  argument, which is optional (with default 0) for an unpaid
  entrypoint.
\item The \lstinline/fee/ parameter is optional in both
  cases. It is precomputed and supplied as the default value in the
  implementation.

  Q (extension): Would it make sense to specify something about fees in the
  contract module? For example, we might specify a fee schedule which
  raises the fee until the call is accepted.
\item The type \lstinline/Tezos.monad/ is defined by the underlying
  API. The untyped API uses a type \lstinline/Answer.t/. The exact
  relationship is TBD.
\item In this example, there are no parameters that are passed to the
  entrypoints because their type is \lstinline/Unit/.
\item Error reporting assumes that the monad returns a sum type. But
  things could be handled differently (for the
  \lstinline/close/ entrypoint): by an error continuation of
  type \lstinline/close_errors -> Answer.t/ or by
  raising an exception that encapsulates the error type:
\begin{lstlisting}
exception Close_Exn of close_errors
\end{lstlisting}
\end{itemize}

If the contract source is available, then an extended type checker verifies that the entry points
exist and that all failures that may arise during contract execution from a particular entry point
are covered by the listed cases.

If the contract source is unavailable, then we can obtain it from the
contract's public hash using an API call. In that case, we would
specify the hash in the contract.
\begin{lstlisting}
sig
  hash "KT1ThEdxfUcWUwqsdergy..."
...
\end{lstlisting}

We propose to treat the remaining errors (i.e., built-in errors
and failures that are not specifed) like Java's unchecked
exceptions. That is, the API function throws an exception.


\begin{lstlisting}[caption={Generated implementation},label={lst:generated-implementation}]
(* constant template *)
open Tezos_api
open Tezos_serialization (* Tint, Tunit, serialize, ... *)

let failwith_regex = 
  "A FAILWITH instruction was reached.*"
let int_regex = 
  "[+-]?[0-9]+"

(* variable part *)
let contract_addr = 
  get_contract "KT1ThEdxfUcWUwqsdergy..." >>=? 
  fun ca -> ???

type bid_errors = 
   | closed       (** auction closed*)
   | too_low      (** bid received is too low *)
   | the_answer   (** failwith received argument 42 *)
   | failedwith of int (** failwith received an integer arg *)

let big_errors_map =
  let open Base in
  Base.Map.of_alist_exn (module String)
    [
      (failwith_regex ^ "closed", closed);
      (failwith_regex ^ "bid too low", too_low);
      (failwith_regex ^ "42", the_answer);
      (failwith_regex ^ int_regex, failedwith (0)); 
      (* badness 10000: actual implementation must obtain the int value *)
    ]
  (* inadequate in several ways;
   * should just generate code instead of using a map*)
  

let bid destination ?(fee=100.0) amount v =
  ...
  begin
    SyncAPIV0.call_contract
      amount
      destination
      contract_addr
      ~(serialize Tint v)
      fee
    >>=? fun oph ->
    wait_for_inclusion oph (* busy waiting *)
    >>=? fun _ ->
    return 0
  end
  >>= function
  | Ok _ -> ...
  | Error (Rejection Michelson_runtime_error s) -> 
    (*match s against expected errors from ...errors_map *)
    (*throw exception if no match *)
    ...

type close_errors = 
   | not_owner    (** caller cannot close the auction *)

let close destination ?(fee=50.0) ?(amount=0.0) v =
  ...
  SyncAPIV0.call_contract
    amount
    destination
    contract_addr
    ~(serialize Tunit v)
    fee
\end{lstlisting}
This code example is incomplete and likely incorrect. It contains
fragments from an example implemented on top of the synchronous API.
The call to \lstinline/serialize/ refers to a proposal to improve on
the typing properties of the synchronous API.

\clearpage
\section{Tools}
\label{sec:tools}

An extended type checker collects information about explicit failures
along execution paths starting from each possible entry point. This
checker will be able to verify the contract module specification in
Listing~\ref{lst:contract-module-example} against the Michelson code
in Listing~\ref{lst:code-example}. We may also wish to provide an
entry point so that the checker only gathers information for that
entry point. This alternative may be interesting if there is an entry
point taking a parameter of sum type, rather than using all injections
to distinguish between entry points.

The checker should warn about unhandled failures (both checked and unchecked).

The recommended technology for the type checker is symbolic
execution. In this framework, it would also 
be possible to specify further invariants (like the
\lstinline/require/ clauses) about the inputs and place
them into the contract module. Essentially, every formula that an SMT
solver can deal with should be fine, although a first version of the
symbolic executor could work without using SMT.

The checker should also make sure that unpaid entrypoints do not
depend on a non-zero amount being passed to the contract
invocation. For example, the \texttt{AMOUNT} instruction of Michelson
should not be used in the code.  

The checker should precalculate the fee. It essentially obtains
this value from the Tezos library.

Failures might arise in contracts that are transitively invoked. In
this case, the checker would obtain the callees' code and process
it. The first step, not too difficult, in this direction would be to
approximate the contents of the list of operations that is produced by
the contract. 

A starting point for the implementation could be the module \lstinline/Tezos_micheline/, which has a
Michelson parser.

\subsection{The Checker}
\label{sec:checker}

\subsubsection{Input}
\begin{itemize}
\item A contract in the form of a \texttt{.tz} file containing a
  Michelson program.
\item An entrypoint in the program
\end{itemize}

The checker performs symbolic execution of the Michelson stack
machine.


A symbolic value is either a concrete value
constructor (like a number, a pair, left, right, nil, cons, etc) where
the components are symbolic values, or it is a Michelson type which
may be associated with a constraint, e.g., for an integer $x$ there 
may be the constraint that $0<x<10$. Subcomponents of the type are
again types with constraints. In a first iteration, constraints may be
restricted to a disjunction of equations, e.g., $x=1 \vee x=2$.

\subsubsection{Type checking}
\label{sec:type-checking}

 The stack is initialized with a symbolic value corresponding to a pair
of the storage type and the parameter type. The latter is concretized
according to the path to the specified entrypoint. That is, to
investigate the entrypoint \lstinline/bid/ in
Listing~\ref{lst:code-example}, symbolic execution would start with a
one-element stack that contains
\begin{lstlisting}[language=michelson]
(PAIR (LEFT x1:int) x2:address)
\end{lstlisting}
without any constraints on \texttt{x1} or \texttt{x2}.

The checker executes instructions as far as possible using the
concrete part and updates the stack according to the typing rules and
semantics of the instruction. It tries to keep the
stack contents as concrete as possible by exploiting all known information
(e.g., constraints). If control flow cannot be decided, the checker
explores all alternatives, potentially adding constraints that reify the
checked condition, and joins the results. Execution paths with
unsatisfiable constraints are pruned.

\subsubsection{Exception tracking}
\label{sec:exception-tracking}

If the checker encounters an instruction that can make the contract
fail, it saves this potential failure along with the (control flow) constraints
leading to it. For a \texttt{FAILWITH} instruction, it saves
this failure along with the constraints and with the symbolic value on
top of the stack.

\subsubsection{Token tracking}
\label{sec:token-tracking}

Instructions that generate tokens (e.g., \texttt{AMOUNT},
\texttt{BALANCE}) create symbolic values marked with 
the instruction. Further processing of these symbolic values leads to
constraints on tokens and to their usage. These are also saved to
infer if an amount argument is needed.

\end{document}

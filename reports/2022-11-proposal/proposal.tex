\documentclass[a4paper,11pt]{article}

\usepackage[utf8]{inputenc}
\usepackage{url}
\usepackage{hyperref}
\usepackage{authblk}

\title{A Statically Checked Interface and a Checker for Tezos Smart Contracts}
\author{Peter Thiemann}
\author{Thi Thu Ha Doan}
\affil{University of Freiburg, Germany}
\begin{document}
\maketitle{}

\section{Executive Summary}
\label{sec:executive-summary}
The Tezos blockchain (like other blockchains) provides a Web-based API and a client library for interacting with smart contracts on the chain. Programmatic interaction with a blockchain is often error prone. Web interfaces handle semi-structured data, often in JSON format, which is cumbersome to use and offers few guarantees. The client library also accepts data in a serialized format. Thus, the type safety of a contract call is not statically guaranteed and bad calls leads to run-time errors.

Even type safe smart contract invocations may fail, if their preconditions are not fulfilled. Such failures are costly because transaction fees are wasted.  Some of these failures can predicted before the invocation by probing the current state of the contract and exploiting known invariants and preconditions. While invariants can be established with a theorem prover, developers who are not experts in this area need automatic approaches to formal verification of  smart contracts. The goals of this research are:

\begin{itemize}
\item to provide statically checked interfaces wrapped in Contract Modules to interact with contracts on the Tezos blockchain. This interface statically checks against the contract implementation to ensure type safety and exception handling (any error condition that occurs is handled by an appropriate error message). One of the most important features is to statically prevent a failed call; 
\item to develop an automatic checker for Michelson contracts to help developers formally specify the orchestration of their contracts and check their smart contracts before deployment.
\end{itemize}

This work is a continuation and extension of the Contract Orchestration for OCaml (COOC) project funded by the Tezos Foundation. The summary of this work can be found on Github \cite{tezos-report}.

\section{Researchers}
\label{sec:researchers}
\paragraph{Peter Thiemann}
is a full professor for Informatics at Freiburg University.
His research focuses on programming and proving with types. Since 2004, he worked on
typing aspects as well as pragmatics of web programming. His contributions include client-server calculus, multi-tier programming, and protocol specialization, which are closely related to session types. With Neubauer, he was one of the first to propose and investigate multi-tier
programming. He served on the program committees of major programming languages conferences (POPL,
ICFP, OOPSLA, ESOP, ECOOP, LICS, CSF), was PC chair for ICFP, ESOP, and PPDP, and is currently chair of the POPL steering committee. From 2015-2018 he served as vice-chair of ACM
SIGPLAN. He is also an associate editor of the Journal of Functional Programming.


\paragraph{Thi Thu Ha Doan} is a postdoctoral researcher at Freiburg University. She is currently employed by the ongoing COOC project. Her research focuses on formal methods, formal verification, type and program, smart contracts, and blockchain. She has been a member of the program committees of APLAS 2022 and ICFEM 2022. 

Doan and Thiemann have been working (with support by several student workers) on the COOC project funded by the Tezos Foundation  since 2020. Thiemann and Doan have worked on various aspects of type-safe interaction, distributed verification of assertions and Contract modules, and a checker for Michelson smart contracts. They proposed a type-safe interaction between an OCaml program and Michelson smart contracts on the Tezos blockchain. The typed OCaml API for Tezos was fully implemented. The work was presented and published at the APLAS 2021 conference \cite{DBLP:conf/aplas/DoanT21}. The initial state of work on Contract modules was presented and published at the Formal Methods for Blockchains (FMBC 2021) workshop \cite{DBLP:conf/cav/Doan021}. 


\section{A Statically Checked Programming Interface}
\label{sec:checked-programming-interfect}

A Michelson contract fails if its execution runs into a \texttt{FAILWITH} statement. In this case, execution stops and the caller is notified with an error message. The transaction fee is still due, so failure is expensive. Verification of a contract avoids such failures by identifying/checking preconditions such that \texttt{FAILWITH} statements are not reachable. While full-blown verification with an interactive theorem prover is an option for sophisticated developers \cite{DBLP:conf/isola/BernardoCCJPT20}, we aim for an automatic solution.

The basic idea is that we consider ``Contract Modules'' that specify pre- and postconditions as well as contract orchestrations, i.e., valid sequences of contract invocations, i.e., uses of contract entrypoints. We generate code to check the preconditions off-chain before the contract is invoked. Starting from a smart contract implementation, we automatically generate a contract module whose specification includes all types as well as information about possible failure conditions of the contract to ensure mostly fail-safe invocation of the contract. We cannot guarantee full fail-safety due to possibly interleaved contract invocations from other parties.  Contract modules provide a high-level, custom generated interface that reduces a contract call to a function call in the language.


To obtain pre- and postconditions, we proposed a simple symbolic execution model for Michelson that allows us to guarantee secure invocations of a contract on the blockchain.
%% The model allows us to formulate preconditions and postconditions, and then specify invariants and constraints that guarantee a  safe invocation. 
 We can also use the model to constrain the possible invocation sequences of entrypoints. The specification is checked against the contract implementation by symbolic execution. When a smart contract is invoked, the invocation can be checked against these invariants and constraints before being used in the chain, where the invocation can succeed. For our FMBC paper \cite{DBLP:conf/cav/Doan021}, we used Michelsym, a simple symbolic interpreter for Michelson. In this continuation project, we want to improve symbolic execution by incorporating an SMT solver like Z3 \cite{DBLP:conf/tacas/MouraB08}. 

\section{A Static Checker for Tezos Smart Contracts}
\label{sec:static-checker}

Verification of a smart contract is important because its
implementation cannot be changed once it is deployed. However,
existing tools \cite{DBLP:conf/isola/BernardoCCJPT20} require a high
level of sophistication in developers 
to formally verify Michelson contracts. Therefore, another goal of
our research  is to develop an automatic verification tool, a checker, for
Michelson smart contracts. When a smart contract script is passed
along with the specification of properties, the tool statically checks
whether these properties are satisfied. We first provide a
domain-specific language (DSL) in which the user can specify the
properties. The property specifications are then converted into
formulas in Z3 in OCaml, which are then checked against the smart
contract implementation using symbolic execution. 

Ultimately, we hope to provide a user-friendly checker for Tezos smart contract developers, where they only need to submit their contract implementation and the properties written in our DSL. The tool will statically check and report any invalid properties.
 

\section{Proposed Timeline and Activities}
\label{sec:prop-timel-activ}

\subsection{January 2023 - June 2023}

\begin{itemize}
\item Activity: Development of a dynamic logic as a formal foundation for symbolic interpretation of Michelson. 
\item Deliverable: Formalization (in Agda) of a dynamic logic for a core fragment of Michelson with soundness proof.
\item Deliverable: Paper about this formalization (writeup starting July 2023).
\end{itemize}

\subsection{July 2023 - December 2023}
\label{sec:july-2023-december}

\begin{itemize}
\item Activity: Implementation of symbolic interpreter for a core fragment of Michelson in OCaml based on an SMT solver, most likely on Z3. 
\item Deliverable: Symbolic interpreter.
\item Deliverable: Case studies with example contracts.
\item Deliverable: Paper or tutorial on this tool (writeup starting January 2024)
\end{itemize}

\subsection{January 2024 - June 2024}
\label{sec:january-2024-june}

\begin{itemize}
\item Activity: Design and implementation of a DSL for static contract properties. This DSL will compile to SMT and interface with the symbolic interpreter to perform verification.  
\item Deliverable: Implementation of the DSL aka contract checker.
\item Deliverable: Report on this tool.
\end{itemize}

\subsection{July 2024 - December 2024}
\label{sec:july-2024-december}

\begin{itemize}
\item Activity: Complete design and implementation of the contract module DSL including contract orchestration. 
\item Deliverable: Full implementation of contract modules.
\item Deliverable: Followup on our FMBC paper. 
\end{itemize}


\section{Allocated Funds}
\label{sec:allocated-funds}

A total of 186,000 EUR distributed as follows:
\begin{itemize}
\item 160,000 EUR: one postdoc position for two years at 80,000 EUR per year.
\item 12,000 EUR: part time student worker position for 12 months in total (help with implementation work).
\item 12,000 EUR: travel funds for conference attendance (four conferences at 3000 EUR per conference). 
\item 2,000 EUR: incidentals like replacement for broken server, broken laptop, etc.
\end{itemize}

\subsection{Remarks}
\label{sec:remarks}

\begin{itemize}
\item
  Travel funds should be applicable to all project participants:
  student workers, postdoc, project lead.
\item 
  Funds for students, travel, and incidentals should be
  interconvertible.
\item 
  The total amount of 186,000 EUR matches approximately the funds
  remaining from the currently running COOC project.
\end{itemize}
\bibliography{report}
\bibliographystyle{plainurl}
\end{document}
